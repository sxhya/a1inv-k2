\documentclass[oneside,12pt]{amsart}
\usepackage{amssymb, amsxtra, amsmath, amstext, amsthm, amsfonts, amscd, hyperref, tikz, graphicx, thmtools}
\usepackage[T2A]{fontenc}
\usepackage[cp1251]{inputenc}
\usepackage[all]{xy}
\usepackage[notref,notcite]{showkeys}
\usepackage[capitalize]{cleveref}
\usetikzlibrary{matrix,arrows}

\oddsidemargin 5mm
\marginparwidth 5mm
\topmargin 0mm
\textheight 225mm
\textwidth 165mm
\headheight 0mm
\headsep 10mm
\footskip 5mm

\newtheorem{thm}{Theorem}
\Crefname{thm}{Theorem}{Theorems}
\numberwithin{equation}{section}

\newtheorem{lem}{Lemma}
\numberwithin{lem}{section}
\Crefname{lemma}{Lemma}{Lemmas}

\newtheorem{cor}[lem]{Corollary}
\Crefname{cor}{Corollary}{Corollaries}

\newtheorem{prop}[lem]{Proposition}
\Crefname{prop}{Proposition}{Propositions}

\newtheorem*{thm*}{Theorem}
\newtheorem*{lemma*}{Lemma}

\theoremstyle{definition}

\newtheorem{dfn}[lem]{Definition}
\Crefname{dfn}{Definition}{Definitions}
\newtheorem{example}[lem]{Example}
\Crefname{example}{Example}{Examples}

\theoremstyle{remark}

\newtheorem{rem}[lem]{Remark}
\Crefname{rem}{Remark}{Remarks}

\DeclareMathOperator{\UU}{U}
\DeclareMathOperator{\LL}{L}
\DeclareMathOperator{\FU}{FU}
\DeclareMathOperator{\GL}{GL}
\DeclareMathOperator{\PGL}{PGL}
\DeclareMathOperator{\Sp}{Sp}
\DeclareMathOperator{\SL}{SL}
\DeclareMathOperator{\St}{St^G}
\DeclareMathOperator{\EE}{E}
\DeclareMathOperator{\OO}{O}
\DeclareMathOperator{\Ep}{\mathcal E}
\DeclareMathOperator{\WW}{W}
\DeclareMathOperator{\HH}{H}
\DeclareMathOperator{\XX}{X}
\DeclareMathOperator{\RR}{\mathcal R}
\DeclareMathOperator{\Lie}{Lie}
\DeclareMathOperator{\Cent}{Cent}
\DeclareMathOperator{\rad}{rad}
\DeclareMathOperator{\Norm}{Norm}
\DeclareMathOperator{\Hom}{Hom}
\DeclareMathOperator{\Spec}{Spec}
\DeclareMathOperator{\Ga}{{\mathbf G}_a}
\DeclareMathOperator{\Gm}{{\mathbf G}_m}
\DeclareMathOperator{\Ker}{ker\,}
\DeclareMathOperator{\coker}{coker\,}
\DeclareMathOperator{\im}{im\,}
\DeclareMathOperator{\Aut}{Aut}
\DeclareMathOperator{\Out}{Out}
\DeclareMathOperator{\End}{End}
\DeclareMathOperator{\Map}{Map\,}
\DeclareMathOperator{\Dyn}{Dyn}
\DeclareMathOperator{\Sym}{Sym}
\DeclareMathOperator{\Gal}{Gal}
\DeclareMathOperator{\lev}{lev}
\DeclareMathOperator{\der}{der}
\DeclareMathOperator{\res}{res}
\DeclareMathOperator{\tr}{tr}
\DeclareMathOperator{\crank}{c-rank}
\DeclareMathOperator{\rank}{rank}
\DeclareMathOperator{\ZZ}{{\mathbb Z}}
\DeclareMathOperator{\QQ}{{\mathbb Q}}
\DeclareMathOperator{\NN}{{\mathbb N}}
\DeclareMathOperator{\PP}{{\mathbb P}}
\DeclareMathOperator{\et}{\text{\it \'et}}
\DeclareMathOperator{\fppf}{\text{\it fppf}}
\DeclareMathOperator{\FF}{{\mathbb F}}
\DeclareMathOperator{\La}{{\mathcal L}}

\newcommand{\id}{\text{\rm id}}
\newcommand{\Aff}{\mathbb {A}}
\newcommand{\Pro}{\mathbb {P}}
\newcommand{\pprime}{\mathfrak{p}}
\newcommand{\ad}{\mathrm{ad}}
\newcommand{\scl}{{sc}}
\newcommand{\ha}{{\widetilde{\alpha}}}
\newcommand{\qs}{\mathrm{qs}}
\newcommand{\eps}{\varepsilon}
\newcommand{\st}{\scriptstyle}
\newcommand{\ds}{\displaystyle}
\newcommand{\xx}{\hbox{${\bf (XX^{-1})}$}\ }
\newcommand{\ee}{\hbox{${\bf (E)}$}}
\newcommand{\Nu}{{\mathrm N}}
\newcommand{\Mu}{{\mathrm M}}
\newcommand{\rA}{\mathsf{A}}
\newcommand{\rB}{\mathsf{B}}
\newcommand{\rC}{\mathsf{C}} 
\newcommand{\rD}{\mathsf{D}} 
\newcommand{\rE}{\mathsf{E}}
\newcommand{\rF}{\mathsf{F}}
\newcommand{\rG}{\mathsf{G}}

\let\l\left
\let\r\right
\let\semir\ltimes
\let\semil\rtimes

\begin{document}

\selectlanguage{english}

\title{$\mathbb{A}^1$-invariance for unstable $K_2$}

\author{S. Sinchuk}
\address{Department of Mathematics and Mechanics, St. Petersburg State University,
St. Petersburg, Russia}
\email{sinchukss@gmail.com}

\author{A. Stavrova}
\address{Department of Mathematics and Mechanics, St. Petersburg State University,
St. Petersburg, Russia}
\email{anastasia.stavrova@gmail.com}

\author{A. Lavrenov}
\address{Mathematisches Institut der Universit\"at M\"unchen, Theresienstr. 39, D-80333 M\"unchen}
\email{avlavrenov@gmail.com}

\subjclass[2010]{19C09, 19C20, 14L15, 20G35}
\keywords{Chevalley group, non-stable $K_2$-functor, Steinberg group}

\maketitle

%\begin{abstract}
%To be written
%\end{abstract}

\section{Aim of the paper}

The aim of this text is to prove that the non-stable $K_2$-functors $K_2^G$, where $G$ is a simply
connected Chevalley group
of suitable type, satisfy $\Aff^1$-invariance on regular rings $R$ containing a field $k$, that is,
\begin{equation}\label{eq:A1-main}
K_2^G(R[t])=K_2^G(R).
\end{equation} As a corollary, we should easily deduce that
$$K_2^G(R)=KV_2^G(R),$$
where $KV_2^G(R)$ is the Karoubi--Villamayor $K$-functor associated to $G$. This functor originates from~\cite{J}.
The above equality by e.g.~\cite[Corollary 4.3.3]{AHW15} implies that $\pi_1^{\Aff^1}(G)(R)=K_2^G(R)$, i.e. we obtain an explicit presentation
for the $\Aff^1$-fundamental group of $G$ in the sense of Morel--Voevodsky.

"Suitable type"{} here means that we consider only the cases where we know the centrality of $K_2$,
or at least the Quillen-Suslin lgp. Some intermediate steps can be proved in larger generality.

Essentially, we need to prove that $K_2^G(k[x_1,\ldots,x_n])=K_2^G(k)$. (Then~\eqref{eq:A1-main} follows by standard geometric methods.)
There are two models: Tulenbaev's proof for the $\SL_n$ case~\cite{Tu} and Stavrova's proof for $K_1^G$~\cite{St-poly}.
Tulenbaev~\cite{Tu} uses stabilization of the $K_2$-functor,
and the good properties of the limit=algebraic $K$-theory.
In~\cite{St-poly} stabilization is not used. However, the key steps
of both proofs are the same: the case of $R=k$ (hidden somewhere around~\cite[p. 140]{Tu}, or, respectively,
~\cite[Theorem 3.1]{St-poly}); Quillen-Suslin lgp; $\Pro^1$-gluing (see~\cite[Theorem 5.1]{Tu} or~\cite[Theorem 1.1]{St-poly}).

\subsection{The case $R=k$}
We consider the case of $K_2^G(k[t])$ vs. $K_2^G(k)$.
(As in the $\Aff^1$-invariance of $K_1^G$, this case should be used to deduce that $K_2^G(k[t_1,\ldots,t_n])=K_2^G(k)$.)

In Tulenbaev's framework, it follows from stabilization.
However, the equality $K_2^G(k[t])=K_2^G(k)$ is sort of known for all groups.
Namely, in~\cite[Theorem 5.1]{W-k[t]}: let $k$ be an infinite field and let $G$ be a connected reductive
group over $k$. Then the inclusion $k\hookrightarrow k[t]$ induces an
isomorphism
$$H_\bullet(G(k),\mathbb{Z})\stackrel{\cong}{\longrightarrow} H_\bullet(G(k[t]),\mathbb{Z}),$$
if the order of the fundamental group of $G$ is invertible in $k$.
Once we know that the homology $H_2$ coincides with $K_2^G$ (on both sides), this gives the result.
It would be nice to check Wendt's proof; maybe, discuss it in a seminar?

Is it necessary to know the centrality in order to show that $K_2^G$ coincides with $H_2$?
If yes, then we probably know it for $K_2^G(k)$ even for isotropic groups~\cite{Deo}, but only for the good
Chevalley groups for $K_2^G(k[t])$.
We should try to understand what is proved in~\cite[Proposition 5.3]{VW} using only the universality of
the Steinberg group. This may be useful. The paper is unpublished, so again everything should be double-checked
if you want to refer to it.

\subsection{Plan of the proof}
\begin{enumerate}
\item\label{plan:3t} ({\bf done}, see \cref{thm:3t}). Let $R$ be a local ring. Show that
$$\St(R[t^{\pm 1}])=i_+(\St(R[t]))i_-(\St(R[t^{-1}]))i_+(\St(R[t]))$$
Here $i_\pm$ denote the natural homomorphisms into $\St(R[t^{\pm 1}])$.

\item\label{plan:k[t]} (open, see \cref{thm:k[t]}). Show that $K_2^G(k[t])=K_2^G(k)$.

\item\label{plan:k-intersect} ({\bf done}).
Consequently, $\St(k[t])\to\St(k[t^{\pm 1}])$ is injective and $\St(k[t])\cap\St(k[t^{-1}])=\St(k)$ inside
$\St(k[t^{\pm 1}])$. Also, $K_2^G(k)=K_2^G(k[t^{\pm 1}])$.

The first claim uses~\eqref{plan:k[t]}. The second claim follows from~\eqref{plan:k[t]} and~\eqref{plan:3t}.

\item\label{plan:QSlgp} ({\bf done} for split ACDE, see \cref{thm:lg-k2}). Prove Quillen-Suslin lgp for $K_2^G$.

\item\label{plan:Zglu} (Zariski gluing) for any commutative ring $A$ and any non-nilpotent $f,g\in A$ such that
$A=fA+gA$, the sequence of pointed sets
$$1\longrightarrow K_2^G(A)\xrightarrow{\st g\mapsto (g,g)} K_2^G(A_f)\times K_2^G(A_g)
\xrightarrow{\st (g_1,g_2)\mapsto g_1{g_2}^{-1}} K_2^G(A_{fg})$$
is exact. The proof is usually almost the same as for~\eqref{plan:QSlgp}.

(Remark. It seems that we need this property only for $A=R[t]$ and $f,g$ non-constant polynomials.)

\item\label{plan:S-lemma} ($S$-lemma)
Let $A$ be a commutative ring, $S$ a multiplicative subset of $A$.
If $$K_2^G(A[X_1,\ldots,X_n])=K_2^G(A)$$ for some $n\ge 1$, then
$K_2^G(A_S[X_1,\ldots,X_n])=K_2^G(A_S)$ as well.

This should be easy; see~\cite[Lemma 3.6]{Abe}.

\item\label{plan:Nglu} (Nisnevich gluing) Assume that $B$ is a subring of a commutative ring $A$, and let
$h\in B$ be a non-nilpotent element. Denote by $F_h:A\to A_h$ the localization homomorphism.

(i) If $Ah+B=A$, i.e. the natural map $B\to A/Ah$ is surjective, then for any $x\in \St(A_h)$ there exist
$y\in \St(A)$ and $z\in \St(B_h)$ such that
$x=F_h(y)z$.

(ii) If moreover $Ah\cap B=Bh$, i.e. $B/Bh\to A/Ah$ is an isomorphism, and $h$ is not a zero divisor in $A$, then
the sequence of pointed sets
$$
K_2^G(B)\xrightarrow{\st g\mapsto (F_h(g),g)} K_2^G(B_h)\times K_2^G(A)\xrightarrow{\st (g_1,g_2)\mapsto g_1F_h(g_2)^{-1}}
K_2^G(A_h)
$$
is exact.

This should use something from the proof of~\eqref{plan:QSlgp} or~\eqref{plan:Zglu}; see~\cite[Lemma 3.4]{St-poly}.

\item\label{plan:P1} ($\Pro^1$-gluing) Let $A$ be any commutative ring. Show that the sequence of pointed sets
$$
1\longrightarrow K_2^G(A)\xrightarrow{\st g\mapsto (g,g)} K_2^G(A[t])\times K_2^G(A[t^{-1}])
\xrightarrow{\st (g_1,g_2)\mapsto g_1{g_2}^{-1}} K_2^G(A[t,t^{-1}])
$$
is exact.

This should use~\eqref{plan:k[t]},~\eqref{plan:QSlgp},~\eqref{plan:3t},~\eqref{plan:k-intersect}.

\item\label{plan:[]f} (Main corollary of $\Pro^1$-gluing) Let $A$ be any commutative ring, and let $f\in A[t]$ be a monic polynomial. Show that
$K_2^G(A[t])\to K_2^G(A[t]_f)$ is injective.

The proof uses~\eqref{plan:Zglu} and~\eqref{plan:P1}.

\item\label{plan:k(t)} Prove that $K_2^G(k(t))=K_2^G(k)$.

This may be a bit tricky; I will think if we can get rid of it. I don't think Tulenbaev uses it.

\item\label{plan:k[tn]} Prove that
$$K_2^G(k[t_1,\ldots,t_n])=K_2^G(k).$$

If we strictly follow the pattern of $K_1^G$, this uses~\eqref{plan:[]f},~\eqref{plan:k[t]}, and
~\eqref{plan:k(t)}. There may be other ways.

\item \label{plan:final} Final result: let $R$ be a regular ring containing a field $k$. Then $K_2^G(R[t])=K_2^G(R)$.

This uses~\eqref{plan:QSlgp},~\eqref{plan:k[tn]} and~\eqref{plan:Nglu}.
\end{enumerate}

%%%%%%%%%%%%%%%%%%%%%%%%%%%%%%%%%%%%%%%%%%%%%%%%%%%%%%%%%%%%%%%%%%%%%%%%%%%%%%%%%%%%%%%%%%%%
%%%%%%%%%%%%%%%%%%%%%%%%%%%%%%%%%%%%%%%%%%%%%%%%%%%%%%%%%%%%%%%%%%%%%%%%%%%%%%%%%%%%%%%%%%%%
\section{Steinberg groups of Chevalley groups: preliminaries}

Definition, functoriality, "congruence subgroups"{} $\St(\Phi,R,I)$ versus $\ker(\St(\Phi,R)\to\St(\Phi,R/I))$.

%The map $\St(R)\to E(R)$ is bijective on subgroups of the form $U_\Psi(R)$, where $\Psi$ is a
%unipotent set of roots.

%%%%%%%%%%%%%%%%%%%%%%%%%%%%%%%%%%%%%%%%%%%%%%%%%%%%%%%%%%%%%%%%%%%%%%%%%%%%%%%%%%%%%%%%%%%%
%%%%%%%%%%%%%%%%%%%%%%%%%%%%%%%%%%%%%%%%%%%%%%%%%%%%%%%%%%%%%%%%%%%%%%%%%%%%%%%%%%%%%%%%%%%%
\section{Decomposition theorems for $\St(A[t^{\pm 1}])$ and $\St\bigl(A((t))\bigr)$.}

\begin{lem}
 Let $(R,m)$ be a local ring, and let $G$ be a simply connected simple group over $R$
of isotropic rank $\ge 2$. Let $i_+:\St(R[t])\to\St(R[t^{\pm 1}])$ and $i_-:\St(R[t^{-1}]\to
\St(R[t^{\pm 1}])$ be the natural homomorphisms.
Then
$$
i_+\bigl(\St(m\cdot R[t])^{\St(R[t])}\bigr)i_-\bigl(\St(R[t^{-1}]\bigr)=i_-\bigl(\St(R[t^{-1}]\bigr)
i_+\bigl(\St(m\cdot R[t])^{\St(R[t])}\bigr)
$$
inside $\St(R[t^{\pm 1}])$.
\end{lem}
\begin{proof}
This is proved exactly as~\cite[Lemma 5.12]{St-poly}.
\end{proof}


\begin{thm}\label{thm:3t}
Let $R$ be a local ring, and let $G$ be a simply connected simple group over $R$
of isotropic rank $\ge 2$. Let $i_+:\St(R[t])\to\St(R[t^{\pm 1}])$ and $i_-:\St(R[t^{-1}]\to
\St(R[t^{\pm 1}])$ be the natural homomorphisms.
Then
$$
\St(R[t^{\pm 1}])=i_+(\St(R[t]))i_-(\St(R[t^{-1}]))i_+(\St(R[t])).
$$
\end{thm}
\begin{proof}
This is proved exactly as~\cite[Theorem 5.1]{St-poly}.
\end{proof}


%%%%%%%%%%%%%%%%%%%%%%%%%%%%%%%%%%%%%%%%%%%%%%%%%%%%%%%%%%%%%%%%%%%%%%%%%%%%%%%%%%%%%%%%%%%%
%%%%%%%%%%%%%%%%%%%%%%%%%%%%%%%%%%%%%%%%%%%%%%%%%%%%%%%%%%%%%%%%%%%%%%%%%%%%%%%%%%%%%%%%%%%%
\section{Quillen-Suslin lgp, Zariski gluing, Nisnevich gluing, $S$-lemma}

\begin{thm}\label{thm:lg-k2}
Let $R$ be arbitrary commutative ring and let $G$ be a simple Chevalley group of type $\rA_\ell, \rC_\ell, \rD_\ell$ or $\rE_\ell$ and rank $\ell \geq 3$. 
An element $g \in \St(R[t], tR[t])$ is trivial if and only if 
its image in $\St(R_M[t], tR_M[t])$ is trivial for all maximal ideals $M \trianglelefteq R$.
\end{thm}
\begin{proof}
The case $\Phi=\rC_\ell$, $\ell\geq 3$ is contained in some future Lavrenov's solo paper.
It think that the case of a simply laced $\Phi$ of rank $\geq 3$ can be settled in a similar way as in \cite{S15}.
We already have a written proof that \cite[Lemma~2.3]{Tu} extends to rank $3$ groups, see~\href{https://github.com/sxhya/LGK2Chevalley/blob/master/paper.tex}{here}.
It should be inserted into this article at a some place.
\end{proof}

%%%%%%%%%%%%%%%%%%%%%%%%%%%%%%%%%%%%%%%%%%%%%%%%%%%%%%%%%%%%%%%%%%%%%%%%%%%%%%%%%%%%%%%%%%%%
%%%%%%%%%%%%%%%%%%%%%%%%%%%%%%%%%%%%%%%%%%%%%%%%%%%%%%%%%%%%%%%%%%%%%%%%%%%%%%%%%%%%%%%%%%%%
\section{The case of $K_2^G(k[t])$ and some corollaries}

\begin{thm}\label{thm:k[t]}
Let $k$ be a field.
Let $G=G(\Phi,-)$ be a simply connected simple Chevalley group of rank $\ge 2$ such that ...
Then $$K_2^G(k[t])=K_2^G(k).$$
\end{thm}
\begin{proof}
???
\end{proof}

\begin{cor}\label{cor:k[t]inj}
Let $G$, $k$ be as in \cref{thm:k[t]}. Then $\St(k[t])\to\St(k[t^{\pm 1}])$ is injective and $\St(k[t])\cap\St(k[t^{-1}])=\St(k)$ inside
$\St(k[t^{\pm 1}])$.
\end{cor}
\begin{proof}
Clearly, $g\in\ker(\St(k[t])\to\St(k[t^{\pm 1}]))$ implies $g\in K_2^G(k[t])$. Since $K_2^G(k[t])=K_2^G(k)$,
and  there is a section $K_2^G(k[t^{\pm 1}])\to K_2^G(k)$, the map is injective.
Second claim: take $g\in \St(k[t])\cap\St(k[t^{-1}])$.
Then the image $\phi(g)$ belongs to $E(k)=E(k[t])\cap E(k[t^{-1}])$, and
after adjusting $g$ by an element of $\St(k)$, we can assume that $g\in K_2^G(k[t])\cap K_2^G(k[t^{-1}])$. Hence
$g\in K_2^G(k)\subseteq\St(k)$.
\end{proof}

\begin{cor}
Let $G$, $k$ be as in \cref{thm:k[t]}. Then $K_2^G(k[t^{\pm 1}])=K_2^G(k)$.
\end{cor}
\begin{proof}
We use \cref{thm:3t}.
Take $g\in K_2^G(k[t^{\pm 1}])$,
then $g=x_1yx_2$, $x_i\in \St(k[t]))$, $y\in \St(k[t^{-1}]))$ .
Since $E(k[t])\cap E(k[t^{-1}])=E(k)$,
we have $y\in\St(k)K_2^G(k[t^{-1}])=\St(k)$ and $x_1x_2\in\St(k)$. That is, $g\in K_2^G(k)$.
\end{proof}

%%%%%%%%%%%%%%%%%%%%%%%%%%%%%%%%%%%%%%%%%%%%%%%%%%%%%%%%%%%%%%%%%%%%%%%%%%%%%%%%%%%%%%%%%%%%
%%%%%%%%%%%%%%%%%%%%%%%%%%%%%%%%%%%%%%%%%%%%%%%%%%%%%%%%%%%%%%%%%%%%%%%%%%%%%%%%%%%%%%%%%%%%
\section{$\Pro^1$-gluing}

The following lemma is an analog of~\cite[Proposition 4.3 (a)]{Tu}.

\begin{lem}
Let $R$ be a local ring, $k=R/m$, and let $k$, $G$ be as in \cref{thm:k[t]}.
The natural homomorphism $\St(R[t])\to\St(R[t^{\pm 1}])$ is injective.
\end{lem}
\begin{proof}
Let $I$ be the maximal ideal of $R$,
$l=R/I$, and consider the natural maps $\rho:\St(R[t,t^{-1}])\to \St(l[t,t^{-1}])$,
$\rho_+:\St(R[t])\to \St(l[t])$, $\rho_-:\St(R[t^{-1}])\to\St(l[t^{-1}])$.
Take $x\in \ker(\St(R[t])\to\St(R[t^{\pm 1}])$.
By the field case \cref{cor:k[t]inj} one has
$\rho_+(x)=1,$ hence $x\in\St(I\cdot R[t])^{\St(R[t])}$.

???????
\end{proof}

\begin{lem}
Let $G$, $k$ be as in \cref{thm:k[t]}.
Let $(R,m)$ be a local ring such that $R/m=k$.
Then
$$\St(R[t])\cap \St(R[t^{-1}])=\St(R)$$
inside $\St(R[t^{\pm 1}])$.
\end{lem}
\begin{proof}
??????
\end{proof}

\begin{thm}
Let $A$ be any commutative ring. Then the sequence of pointed sets
$$1\longrightarrow K_2^G(A)\xrightarrow{\st g\mapsto (g,g)} K_2^G(A[t])\times K_2^G(A[t^{-1}])
\xrightarrow{\st (g_1,g_2)\mapsto g_1{g_2}^{-1}} K_2^G(A[t,t^{-1}])$$
is exact.
\end{thm}
\begin{proof}
????
\end{proof}

\begin{cor}
 Let $A$ be any commutative ring, and let $f\in A[t]$ be a monic polynomial. Show that
$K_2^G(A[t])\to K_2^G(A[t]_f)$ is injective.
\end{cor}
\begin{proof}
????
\end{proof}

\renewcommand{\refname}{References}
\begin{thebibliography}{MMMM}

\bibitem[A]{Abe} E. Abe, 
{\it Whitehead groups of Chevalley groups over polynomial rings},
Comm. Algebra {\bf 11} (1983), 1271--1307.

\bibitem[AHW]{AHW15} A. Asok, M. Hoyois, M. Wendt, 
{\it Affine representability results in $A^1$-homotopy theory II: principal bundles and homogeneous spaces},
2015~\href{http://arxiv.org/abs/1507.08020}{arXiv:1507.08020}.

\bibitem[Deo]{Deo} V. V. Deodhar,
{\it On central extensions of rational points of algebraic groups}, 
Amer. J. Math. {\bf 100} (1978), 303--386.

\bibitem[J]{J} J.F. Jardine, 
{\it On the homotopy groups of algebraic groups}, 
J. Algebra {\bf 81} (1983), 180--201.

\bibitem[S15]{S15} S. Sinchuk,
{\it \href{http://dx.doi.org/10.1016/j.jpaa.2015.08.003}{On centrality of $\mathrm{K}_2$ for {C}hevalley groups of type $\rE_\ell$}},
J. Pure Appl. Algebra {\bf 220} (2016), 857--875.

\bibitem[St14]{St-poly} A. Stavrova, 
{\it Homotopy invariance of non-stable $K_1$-functors}, 
J. K-Theory {\bf 13} (2014), 199--248.

\bibitem[St15]{St-serr} A. Stavrova, 
{\it \href{http://dx.doi.org/10.4153/CJM-2015-035-2}{Non-stable $K_1$-functors of multiloop groups}},
Canad. J. Math. (2015), Online First.

\bibitem[Su]{Sus} A.A.~Suslin,
{\it On the structure of the special linear group over polynomial rings},
Math. USSR Izv. {\bf 11} (1977), 221--238.

\bibitem[Tul82]{Tu}
M.~S. Tulenbaev, \emph{The {S}teinberg group of a polynomial ring}, 
Mat. Sb. (N.S.) \textbf{117(159)} (1982), no.~1, 131--144.

\bibitem[Wen14]{W-k[t]}
M.~Wendt, \emph{On homology of linear groups over {$k[t]$}},
Math. Res. Lett. \textbf{21} (2014), no.~6, 1483--1500.

\bibitem[VW]{VW} K. V\"olkel, M. Wendt, 
{\it On $\Aff^1$-fundamental groups of isotropic reductive groups},
2012~\href{http://arxiv.org/abs/1207.2364}{arXiv:1207.2364}.

\end{thebibliography}

\end{document}
