\documentclass[oneside,12pt]{amsart}
\usepackage{amssymb, amsxtra, amsmath, amstext, amsthm, amsfonts, amscd, enumitem, thmtools, tikz, graphicx, tikz-cd}
\usepackage[breaklinks=true, pdfencoding=auto]{hyperref}
\usepackage[cp1251]{inputenc}
\usepackage[all]{xy}
\usepackage[notref,notcite]{showkeys}
\usepackage[capitalize]{cleveref}
\usepackage[toc,page]{appendix}
\usepackage[english]{babel}
\usetikzlibrary{matrix,arrows,cd}

\definecolor{darkblue}{rgb}{0.0, 0.0, 0.6}

\hypersetup{colorlinks=true, urlcolor=darkblue, linkcolor=darkblue, citecolor=darkblue}

\DeclareRobustCommand{\VAN}[2]{#1}
\usepackage[hyperref=true, backend=biber, citestyle=numeric-comp, sortlocale=en_US, url=false, doi=false, eprint=true, firstinits=true, maxbibnames=4]{biblatex}
\addbibresource{a1inv-k2.bib}
\renewbibmacro*{volume+number+eid}{\ifentrytype{article}{\- \iffieldundef{volume}{}{{\bf\printfield{volume}},}\iffieldundef{number}{}{ no.~\printfield{number},}}}
\renewbibmacro{in:}{\ifentrytype{article}{}{\printtext{\bibstring{in}\intitlepunct}}}
\newbibmacro{string+doi}[1]{\iffieldundef{doi}{\iffieldundef{url}{#1}{\href{\thefield{url}}{#1}}}{\href{http://dx.doi.org/\thefield{doi}}{#1}}}
\DeclareFieldFormat{title}{\usebibmacro{string+doi}{\mkbibemph{#1}}}
\DeclareFieldFormat[article, inproceedings, inbook, thesis]{title}{\usebibmacro{string+doi}{\mkbibquote{#1}}}
\DeclareFieldFormat[online]{title}{``#1''}
\renewcommand*{\bibfont}{\footnotesize}

\oddsidemargin 5mm
\marginparwidth 5mm
\topmargin 0mm
\textheight 225mm
\textwidth 165mm
\headheight 0mm
\headsep 10mm
\footskip 5mm

\newlist{thmlist}{enumerate}{1} \setlist[thmlist]{label=(\roman{thmlisti}), ref=\thethm.(\roman{thmlisti}),noitemsep} \Crefname{thmlisti}{Theorem}{Theorems}
\newlist{proplist}{enumerate}{1} \setlist[proplist]{label=(\roman{proplisti}), ref=\thethm.(\roman{proplisti}),noitemsep} \Crefname{thmlisti}{Proposition}{Propositions}
\newlist{lemlist}{enumerate}{1} \setlist[lemlist]{label=(\roman{lemlisti}), ref=\thelem.(\roman{lemlisti}),noitemsep} \Crefname{lemlisti}{Lemma}{Lemmas}

\newtheorem{thm}{Theorem}
\Crefname{thm}{Theorem}{Theorems}
\numberwithin{equation}{section}

\newtheorem{lem}{Lemma}
\numberwithin{lem}{section}
\Crefname{lemma}{Lemma}{Lemmas}

\newtheorem{cor}[lem]{Corollary}
\Crefname{cor}{Corollary}{Corollaries}

\newtheorem{prop}[lem]{Proposition}
\Crefname{prop}{Proposition}{Propositions}

\newtheorem*{thm*}{Theorem}
\newtheorem*{lemma*}{Lemma}

\theoremstyle{definition}

\newtheorem{dfn}[lem]{Definition}
\Crefname{dfn}{Definition}{Definitions}
\newtheorem{example}[lem]{Example}
\Crefname{example}{Example}{Examples}

\theoremstyle{remark}

\newtheorem{rem}[lem]{Remark}
\Crefname{rem}{Remark}{Remarks}

\DeclareMathOperator{\UU}{U}
\DeclareMathOperator{\LL}{L}
\DeclareMathOperator{\FU}{FU}
\DeclareMathOperator{\GL}{GL}
\DeclareMathOperator{\PGL}{PGL}
\DeclareMathOperator{\Sp}{Sp}
\DeclareMathOperator{\SL}{SL}
\DeclareMathOperator{\St}{St^G}
\DeclareMathOperator{\EE}{E}
\DeclareMathOperator{\OO}{O}
\DeclareMathOperator{\Ep}{\mathcal E}
\DeclareMathOperator{\WW}{W}
\DeclareMathOperator{\HH}{H}
\DeclareMathOperator{\XX}{X}
\DeclareMathOperator{\RR}{\mathcal R}
\DeclareMathOperator{\Lie}{Lie}
\DeclareMathOperator{\Cent}{Cent}
\DeclareMathOperator{\rad}{rad}
\DeclareMathOperator{\Norm}{Norm}
\DeclareMathOperator{\Hom}{Hom}
\DeclareMathOperator{\Spec}{Spec}
\DeclareMathOperator{\Ga}{{\mathbf G}_a}
\DeclareMathOperator{\Gm}{{\mathbf G}_m}
%\DeclareMathOperator{\ker}{ker\,}
\DeclareMathOperator{\Bigker}{Ker\,}
\DeclareMathOperator{\coker}{coker\,}
\DeclareMathOperator{\im}{Im\,}
\DeclareMathOperator{\Aut}{Aut}
\DeclareMathOperator{\Out}{Out}
\DeclareMathOperator{\End}{End}
\DeclareMathOperator{\Map}{Map\,}
\DeclareMathOperator{\Dyn}{Dyn}
\DeclareMathOperator{\Sym}{Sym}
\DeclareMathOperator{\Gal}{Gal}
\DeclareMathOperator{\lev}{lev}
\DeclareMathOperator{\der}{der}
\DeclareMathOperator{\res}{res}
\DeclareMathOperator{\tr}{tr}
\DeclareMathOperator{\crank}{c-rank}
\DeclareMathOperator{\rank}{rank}
\DeclareMathOperator{\ZZ}{{\mathbb Z}}
\DeclareMathOperator{\QQ}{{\mathbb Q}}
\DeclareMathOperator{\NN}{{\mathbb N}}
\DeclareMathOperator{\PP}{{\mathbb P}}
\DeclareMathOperator{\et}{\text{\it \'et}}
\DeclareMathOperator{\fppf}{\text{\it fppf}}
\DeclareMathOperator{\FF}{{\mathbb F}}
\DeclareMathOperator{\La}{{\mathcal L}}

\newcommand{\catname}[1]{{\normalfont\textbf{#1}}}
\newcommand{\id}{\mathrm{id}}
\newcommand{\Stb}{\mathrm{St}}
\newcommand{\Aff}{\mathbb{A}}
\newcommand{\Pro}{\mathbb{P}}
\newcommand{\pprime}{\mathfrak{p}}
\newcommand{\ad}{\mathrm{ad}}
\newcommand{\scl}{{sc}}
\newcommand{\ha}{{\widetilde{\alpha}}}
\newcommand{\qs}{\mathrm{qs}}
\newcommand{\eps}{\varepsilon}
\newcommand{\st}{\scriptstyle}
\newcommand{\ds}{\displaystyle}
\newcommand{\xx}{\hbox{${\bf (XX^{-1})}$}\ }
\newcommand{\ee}{\hbox{${\bf (E)}$}}
\newcommand{\Nu}{{\mathrm N}}
\newcommand{\Mu}{{\mathrm M}}
\newcommand{\rA}{\mathsf{A}}
\newcommand{\rB}{\mathsf{B}}
\newcommand{\rC}{\mathsf{C}}
\newcommand{\rD}{\mathsf{D}}
\newcommand{\rE}{\mathsf{E}}
\newcommand{\rF}{\mathsf{F}}
\newcommand{\rG}{\mathsf{G}}

\newcommand{\eval}[4]{ev_{\scriptstyle \left[\frac{#2[#1] \rightarrow #3}{#1 \mapsto #4}\right]}}
\newcommand{\ev}[3]{\eval{t}{#1}{#2}{#3}}

%stavrova

\let\l\left
\let\r\right
\let\semir\ltimes
\let\semil\rtimes
\DeclareMathOperator{\Stsym}{Sym}
\DeclareMathOperator{\Stsymt}{{Sym}^t}

%stavrova
\begin{document}

\selectlanguage{english}

\title{$\mathbb{A}^1$-invariance for unstable $K_2$}

\author{S. Sinchuk}
\address{Department of Mathematics and Mechanics, St. Petersburg State University, St. Petersburg, Russia}
\email{sinchukss@gmail.com}

\author{A. Stavrova}
\address{Department of Mathematics and Mechanics, St. Petersburg State University, St. Petersburg, Russia}
\email{anastasia.stavrova@gmail.com}

\author{A. Lavrenov}
\address{Mathematisches Institut der Universit\"at M\"unchen, Theresienstr. 39, D-80333 M\"unchen}
\email{avlavrenov@gmail.com}

\subjclass[2010]{19C09, 19C20, 14L15, 20G35}
\keywords{Chevalley group, non-stable $K_2$-functor, Steinberg group}

\maketitle

%\begin{abstract}
%To be written
%\end{abstract}

\section{Aim of the paper}

The aim of this text is to prove that the non-stable $K_2$-functors $K_2^G$, where $G$ is a simply
connected Chevalley group
of suitable type, satisfy $\Aff^1$-invariance on regular rings $R$ containing a field $k$, that is,
\begin{equation}\label{eq:A1-main}
K_2^G(R[t])=K_2^G(R).
\end{equation} As a corollary, we should easily deduce that
$$K_2^G(R)=KV_2^G(R),$$
where $KV_2^G(R)$ is the Karoubi--Villamayor $K$-functor associated to $G$. This functor originates from~\cite{J}.
The above equality by e.g.~\cite[Corollary 4.3.3]{AHW15} implies that $\pi_1^{\Aff^1}(G)(R)=K_2^G(R)$, i.e. we obtain an explicit presentation
for the $\Aff^1$-fundamental group of $G$ in the sense of Morel--Voevodsky.

"Suitable type"{} here means that we consider only the cases where we know the centrality of $K_2$,
or at least the Quillen-Suslin lgp. Some intermediate steps can be proved in larger generality.

Essentially, we need to prove that $K_2^G(k[x_1,\ldots,x_n])=K_2^G(k)$. (Then~\eqref{eq:A1-main} follows by standard geometric methods.)
There are two models: Tulenbaev's proof for the $\SL_n$ case~\cite{Tu} and Stavrova's proof for $K_1^G$~\cite{St-poly}.
Tulenbaev~\cite{Tu} uses stabilization of the $K_2$-functor,
and the good properties of the limit=algebraic $K$-theory.
In~\cite{St-poly} stabilization is not used. However, the key steps
of both proofs are the same: the case of $R=k$ (hidden somewhere around~\cite[p. 140]{Tu}, or, respectively,
~\cite[Theorem 3.1]{St-poly}); Quillen-Suslin lgp; $\Pro^1$-gluing (see~\cite[Theorem 5.1]{Tu} or~\cite[Theorem 1.1]{St-poly}).


\subsection{Plan of the proof}
\begin{enumerate}
\item\label{plan:3t} ({\bf done}, see \cref{thm:3t}). Let $R$ be a local ring. Show that
$$\St(R[t^{\pm 1}])=i_+(\St(R[t]))i_-(\St(R[t^{-1}]))i_+(\St(R[t]))$$
Here $i_\pm$ denote the natural homomorphisms into $\St(R[t^{\pm 1}])$.

\item\label{plan:k[t]} ({\bf done}, see \cref{thm:k[t]}). Show that $K_2^G(k[t])=K_2^G(k)$.

\item\label{plan:k-intersect} ({\bf done}).
Consequently, $\St(k[t])\to\St(k[t^{\pm 1}])$ is injective and $\St(k[t])\cap\St(k[t^{-1}])=\St(k)$ inside
$\St(k[t^{\pm 1}])$. Also, $K_2^G(k)=K_2^G(k[t^{\pm 1}])$.

The first claim uses~\eqref{plan:k[t]}. The second claim follows from~\eqref{plan:k[t]} and~\eqref{plan:3t}.

\item\label{plan:QSlgp} ({\bf done} for split ACDE, see \cref{thm:lg-k2}). Prove Quillen-Suslin lgp for $K_2^G$.

\item\label{plan:Zglu} (Zariski gluing) for any commutative ring $A$ and any non-nilpotent $f,g\in A$ such that
$A=fA+gA$, the sequence of pointed sets
$$1\longrightarrow K_2^G(A)\xrightarrow{\st g\mapsto (g,g)} K_2^G(A_f)\times K_2^G(A_g)
\xrightarrow{\st (g_1,g_2)\mapsto g_1{g_2}^{-1}} K_2^G(A_{fg})$$
is exact. The proof is usually almost the same as for~\eqref{plan:QSlgp}.

(Remark. It seems that we need this property only for $A=R[t]$ and $f,g$ non-constant polynomials.)

\item\label{plan:S-lemma} ($S$-lemma)
Let $A$ be a commutative ring, $S$ a multiplicative subset of $A$.
If $$K_2^G(A[X_1,\ldots,X_n])=K_2^G(A)$$ for some $n\ge 1$, then
$K_2^G(A_S[X_1,\ldots,X_n])=K_2^G(A_S)$ as well.

This should be easy; see~\cite[Lemma 3.6]{Abe}.

\item\label{plan:Nglu} (Nisnevich gluing) Assume that $B$ is a subring of a commutative ring $A$, and let
$h\in B$ be a non-nilpotent element. Denote by $F_h:A\to A_h$ the localization homomorphism.

(i) If $Ah+B=A$, i.e. the natural map $B\to A/Ah$ is surjective, then for any $x\in \St(A_h)$ there exist
$y\in \St(A)$ and $z\in \St(B_h)$ such that
$x=F_h(y)z$.

(ii) If moreover $Ah\cap B=Bh$, i.e. $B/Bh\to A/Ah$ is an isomorphism, and $h$ is not a zero divisor in $A$, then
the sequence of pointed sets
$$
K_2^G(B)\xrightarrow{\st g\mapsto (F_h(g),g)} K_2^G(B_h)\times K_2^G(A)\xrightarrow{\st (g_1,g_2)\mapsto g_1F_h(g_2)^{-1}}
K_2^G(A_h)
$$
is exact.

This should use something from the proof of~\eqref{plan:QSlgp} or~\eqref{plan:Zglu}; see~\cite[Lemma 3.4]{St-poly}.

\item\label{plan:P1} ($\Pro^1$-gluing) Let $A$ be any commutative ring. Show that the sequence of pointed sets
$$
1\longrightarrow K_2^G(A)\xrightarrow{\st g\mapsto (g,g)} K_2^G(A[t])\times K_2^G(A[t^{-1}])
\xrightarrow{\st (g_1,g_2)\mapsto g_1{g_2}^{-1}} K_2^G(A[t,t^{-1}])
$$
is exact.

This should use~\eqref{plan:3t}--\eqref{plan:QSlgp}.

\begin{enumerate}
 \item Prove \cref{prop:kersurj}. This is hard. Have no idea how this can be proved at the moment.
 \item Prove $\Pro^1$-glueing using all the above facts.
\end{enumerate}

\item\label{plan:[]f} (Main corollary of $\Pro^1$-gluing) Let $A$ be any commutative ring, and let $f\in A[t]$ be a monic polynomial. Show that
$K_2^G(A[t])\to K_2^G(A[t]_f)$ is injective.

The proof uses~\eqref{plan:Zglu} and~\eqref{plan:P1}.

\item\label{plan:k(t)} Prove that $K_2^G(k(t))=K_2^G(k)$. This may be a bit tricky; I will think if we can get rid of it. I don't think Tulenbaev uses it.
Unsure whether this is true (?). This is true for $SK_1$ but in view of Milnor's theorem can not hold for $K_1$ and $K_2$.

\item\label{plan:k[tn]} Prove that
$$K_2^G(k[t_1,\ldots,t_n])=K_2^G(k).$$

If we strictly follow the pattern of $K_1^G$, this uses~\eqref{plan:[]f},~\eqref{plan:k[t]}, and
~\eqref{plan:k(t)}. There may be other ways.

\item \label{plan:final} Final result: let $R$ be a regular ring containing a field $k$. Then $K_2^G(R[t])=K_2^G(R)$.

This uses~\eqref{plan:QSlgp},~\eqref{plan:k[tn]} and~\eqref{plan:Nglu}.
\end{enumerate}

%%%%%%%%%%%%%%%%%%%%%%%%%%%%%%%%%%%%%%%%%%%%%%%%%%%%%%%%%%%%%%%%%%%%%%%%%%%%%%%%%%%%%%%%%%%%
%%%%%%%%%%%%%%%%%%%%%%%%%%%%%%%%%%%%%%%%%%%%%%%%%%%%%%%%%%%%%%%%%%%%%%%%%%%%%%%%%%%%%%%%%%%%
\section{Steinberg groups: preliminaries}

%TODO: Definition, functoriality, "congruence subgroups"{} $\St(\Phi,R,I)$ versus $\ker(\St(\Phi,R)\to\St(\Phi,R/I))$.

\subsection{Definition and basic properties}

In what follows $\Phi$ denotes a reduced irreducible root system and $\Pi\subseteq \Phi$ denotes its basis (i.e. the set of simple roots).
Denote by $\widetilde{\alpha}$, $\Phi^+$ and $\Phi^-$, respectively, the maximal root of $\Phi$ and the subsets of positive and negative roots of $\Phi$.
The Dynkin diagram and the extended Dynkin diagram of $\Phi$ corresponding to $\Pi$ will be denoted by $D(\Phi)$, $\widetilde{D}(\Phi)$, respectively.

A proper closed root subset $S\subseteq \Phi$ is called {\it parabolic} (resp. {\it reductive}, resp. {\it special}) if $\Phi=S \cup -S$ (resp. $S = -S$, resp. $S \cap -S=\varnothing$).
Any parabolic subset $S \subseteq \Phi$ can be decomposed into the disjoint union of its reductive and special part, i.e. $S = \Sigma_S \sqcup \Delta_S$, where $\Sigma_S \cap (-\Sigma_S) = \varnothing$, $\Delta_S = -\Delta_S$.

Denote by $m_\beta(\alpha)$ the coefficient of $\beta$ in the expansion of $\alpha$ in $\Pi$, i.\,e. $\alpha = \sum_{\beta\in\Pi} m_\beta(\alpha) \beta$.
For $\beta\in\Pi$ denote by $\Delta_\beta$ the subsystem of $\Phi$ spanned by all simple roots except $\beta$ and by $\Sigma_\beta$ the set consisting of roots $\alpha\in \Phi$ such that $m_\beta(\alpha)>0$.

We denote by $(\alpha, \beta)$ the scalar product of roots and by $\langle \beta, \alpha\rangle$ the integer $2(\beta, \alpha)/(\alpha, \alpha)$.
The Weyl group $W(\Phi)$ is a subgroup of isometries of $\Phi$ generated by all reflections $\sigma_\alpha$, where $\sigma_\alpha(\beta)=\beta-\langle\beta,\alpha \rangle\cdot \alpha$.
For a subset of roots $S \subseteq \Phi$ we denote by $\langle S \rangle$ the root subsystem spanned by $S$, i.e. the minimal subset of $\Phi$ containing $S$ and invariant under the action of reflections $\sigma_\alpha$, $\alpha\in S$.

Let $G=G(\Phi,-)$ be a (split) simply connected simple Chevalley--Demazure group scheme
over $R$ with a root system $\Phi$ of rank $\geq 2$.
Recall that the \emph{Steinberg group} $\St(R)$ (also denoted $\Stb(\Phi, R)$) is defined by means of generators
$\mathcal{X}_{\Phi, R} = \{x_{\alpha}(\xi) \mid \xi\in R, \alpha\in\Phi\}$ and the set of relations $\mathcal{R}_{\Phi, R}$ defined as follows:
\begin{align}
& \phantom{[}
x_\alpha(s) x_\alpha(t) = x_\alpha(s+t),\ \alpha\in\Phi,\ s,t\in R; \label{rel:add}\\
& [x_\alpha(s), x_\beta(t)] = \prod\limits_{i,j\in\mathbb{N}}
 x_{i\alpha + j\beta}\left(N_{\alpha\beta ij}\, s^i t^j\right),\quad \alpha,\beta\in\Phi,\ \alpha\neq\pm\beta,\ s,t\in R. \label{rel:CCF}
\end{align}
The indices $i$, $j$ appearing in the right-hand side of the above relation range over
all positive natural numbers such that $i\alpha + j\beta\in\Phi$.
The structure constants $N_{\alpha \beta i j}=\pm 1,2,3$ appearing in \eqref{rel:CCF} depend only on $\Phi$ and can be computed precisely.

Recall that for $\alpha\in\Phi$, $\varepsilon\in R^*$ the semisimple root elements $h_\alpha(\varepsilon)$ are defined as $h_\alpha(\varepsilon)=w_\alpha(\varepsilon)w_\alpha(-1)$.
Denote by $\WW(\Phi, R)$ the subgroup of $\Stb(\Phi, R)$ generated by all elements
$w_\alpha(\varepsilon)$, $\varepsilon\in R^*$, $\alpha\in\Phi$, and by $\HH(\Phi,R)$
 the subgroup generated by all elements
$h_\alpha(\varepsilon)$, $\varepsilon\in R^*$, $\alpha\in\Phi$.

Following~\cite{Ste73}, we set
$$
\{u,v\}_\alpha=h_\alpha(uv)h_\alpha(u)^{-1}h_\alpha(v)^{-1},\qquad u,v\in R^*,\ \alpha\in\Phi,
$$
and call these elements~\emph{symbols in $\Stb(\Phi,R)$}.
For any ideal $I$ of $R$, we set
$$
\Stsym(\Phi,R,I)=\left<\{u,v\}_\alpha,\ u\in R^*,\ v\in (1+I)^*,\ \alpha\in\Phi\right>\le\Stb(\Phi,R).
$$
The group $\Stsym(\Phi,R,R)$ is denoted by $\Stsym(\Phi,R)$.
Clearly, one has $\{u,v\}_\alpha\in K_2(\Phi,R)$ for any $u,v\in R^*,\ \alpha\in\Phi$. By~\cite[Prop. 1.3 (c)]{Ste73}
the group $\Stsym(\Phi,R,I)$ is  generated by all symbols $\{u,v\}_\alpha$,
$u\in R^*,\ v\in (1+I)^*$, for any fixed long root $\alpha\in\Phi$.


%The map $\St(R)\to E(R)$ is bijective on subgroups of the form $U_\Psi(R)$, where $\Psi$ is a unipotent set of roots.

%%%%%%%%%%%%%%%%%%%%%%%%%%%%%%%%%%%%%%%%%%%%%%%%%%%%%%%%%%%%%%%%%%%%%

\subsection{Non-standard generation of Steinberg groups}

%%%%%%%%%%%%%%%%%%%%%%%%%%%%%%%%%%%%%%%%%%%%%%%%%%%%%%%%%%%%%%%%%%%%%

\begin{lem}\label{lem:parab-gen}
Let $R$ be any commutative ring. Let $\Phi$ be an irreducible root system of rank $\ge 2$, let $\Pi$ be a system of simple roots in $\Phi$,
and let $J\subseteq\Pi$ be such that $|J|\ge 2$. Set
$$
\alpha_J=\sum\limits_{\beta\in J}m_\beta(\alpha)\beta\ \mbox{for any}\ \alpha\in\Phi,
$$
and
$$
\Sigma_J=\{\alpha\in\Phi\ |\ m_\beta(\alpha)>0\ \mbox{for at least one}\ \beta\in J\}.
$$
Let $H$ be the group defined by the generators $x_\alpha(u)$, $\alpha\in\Sigma_J\cup(-\Sigma_J)$,
$u\in R$, and the relations~\eqref{rel:add} and~\eqref{rel:CCF} ranging only over
$\alpha\in\Sigma_J\cup(-\Sigma_J)$, and
$\beta\in\Sigma_J\cup(-\Sigma_J)$ such that $m\alpha_J\neq-k\beta_J$ for all $m,k\in\mathbb{N}$.
Then the natural homomorphism $H\to\Stb(\Phi,R)$ is surjective
and has central kernel. In particular, if $\Stb(\Phi,R)$ is centrally closed, then $H\cong\Stb(\Phi,R)$.
\end{lem}
\begin{proof}
The group scheme $G=G(\Phi,-)$ over $R$ contais two opposite parabolic $R$-subgroups $P^\pm$ such that
$\pm\Sigma_J$ are the sets of roots corresponding to the unipotent radicals of $P^\pm$. One can show that
$H=\Stb_{P^+}(R)$ and $\Stb(\Phi,R)=\Stb_B(R)$ in the sense of~\cite{St-cong}.
% for compatibility of CCF use isomorphism onto unipotent subgroups in E
 By~\cite[Lemma 8]{St-cong}
the natural homomorphism $H\to\Stb(\Phi,R)$ is surjective. By~\cite[Lemma 14]{St-cong} its kernel is central.

\end{proof}



%%%%%%%%%%%%%%%%%%%%%%%%%%%%%%%%%%%%%%%%%%%%%%%%%%%%%%%%%%%%%%%%%%%%%

\subsection{Relative Steinberg groups}

%%%%%%%%%%%%%%%%%%%%%%%%%%%%%%%%%%%%%%%%%%%%%%%%%%%%%%%%%%%%%%%%%%%%%


Denote by $D(R, I)$ the double of the ring $R$ relative to an ideal $I$, i.\,e. the fibered product of rings
$R \times_{R/I} R$ with the natural projections $p_1, p_2 \colon D(R, I) \to R$ defined by $p_i(\xi_1, \xi_2) = \xi_i$, $i=1,2$.
Denote by $G_i$ the kernel of the map $p_i^*\colon\St(D(R, I)) \to \St(R)$.
We define the relative Steinberg group $\St(R, I)$ as $G_1 / C$, where $C = [G_1, G_2]$.
Clearly, there is an exact sequence.
\begin{equation} \label{eq:suite} \begin{tikzcd} 1 \ar[r] & (G_1\cap G_2)/C \ar[r] & \St(R, I) \ar[r, "\overline{p_2^*}"] & \St(R) \ar[r, "\pi^*"] & \St(R/I) \ar[r] & 1 \end{tikzcd} \end{equation}

\begin{lem}
 Assume that $R$ and $I$ are such that the canonical projection $ R \to R/I $ splits.
 Then the following facts are true.
 \begin{lemlist}
   \item \label{item:st-inj} The map $\St(R, I) \rightarrow \St(R)$ is an injection.
   \item \label{item:st-semi} The group $\St(R)$ is isomorphic to $\St(R/I) \ltimes \St(R, I)$.
 \end{lemlist}
\end{lem}
\begin{proof}
 For the proof of the first assertion see \cite[Lemma~8]{S15}.
 Since the group $(G_1 \cap G_2)/C$ vanishes, the sequence \eqref{eq:suite} turns into a split short exact sequence which implies the second assertion.
\end{proof}

\begin{lem} \label{lem:Zgen}
 Let $\Sigma$ be the special subset of some parabolic subset of roots $S \subseteq \Phi$.
 Then the relative Steinberg group $\St(R, I)$ admits the following generating set:
 \[ \mathcal{Z}(\Sigma, R, I) =  \{ x_\alpha(0,s) \cdot C \mid s\in I, \alpha \in \Phi \} \cup \{ z_\alpha(s, \xi) \mid s\in I, \ \xi \in R, \ \alpha \in \Sigma\},\]
 where $z_\alpha(s, \xi)$ denotes the element $x_\alpha(0, s)^{x_{-\alpha}(\xi, \xi)} \cdot C$.
\end{lem}
\begin{proof}
 See \cite[Lemma~5]{S15}.
\end{proof}

\subsection{Tulenbaev's lifting property and its corollaries}
Throughout this section $I \trianglelefteq A$ is an ideal of arbitrary commutative ring $A$.
For a nonnilpotent element $a \in A$ denote by $\lambda_a\colon A \rightarrow A_a$ the morphism of principal localization at $a$.
Consider the following commutative square.
\begin{equation} \label{msq}
 \begin{tikzcd}
    A \ar[r, "\lambda_a"] \ar[d, twoheadrightarrow] \ar[r] & A_a \ar[d, twoheadrightarrow] \\
    A/I \ar[r, "\overline{\lambda_a}"]& A_a/I_a\\
   \end{tikzcd}
\end{equation}
Notice that \eqref{msq} is a pull-back square if and only if $\lambda_a$ induces an isomorphism of $I$ and $I_a$.
Such squares are usually called \emph{Milnor squares} in the literature, see \cite[Ch.~I, \S~2]{Kbook}.

The following property of linear Steinberg groups was discovered for the first time by Tulenbaev
(see~\cite[Lemmas 2.3, 3.2]{Tu}) and plays a key role in the sequel.
\begin{dfn} \label{def:tlp}
We say that the Steinberg group functor $\St$ satisfies {\it Tulenbaev's lifting property}
if for every pull-back square \eqref{msq} the following lifting problem has a solution.
\[\begin{tikzcd} \St(A,   I) \arrow[r, "\mu"] \arrow[d]          &  \St(A) \ar[d, "\lambda_a^*"] \\
                 \St(A_a, I) \arrow[r, "\mu"] \arrow[ur, dotted] &  \St(A_a) \end{tikzcd}\]
\end{dfn}

\begin{thm} Assume that $G$ satisfies Tulenbaev property \eqref{def:tlp} then the following facts are true for arbitrary commutative ring $A$:
\begin{thmlist}
 \item \label{thm:dp} A dilation principle holds for $\St(-)$, i.\,e. if $g\in\St(A[t], tA[t])$ is such that equality $\lambda_a^*(h) = 1$ holds in $\St(\Phi, R_a[t])$ then
       for sufficiently large $n$ one has $$\ev{R}{R[t]}{a^n\cdot t}^*(h) = 1.$$
 \item \label{thm:lg-k2} A local-global principle holds for $\St(-)$, i.\,e. an element $g \in \St(A[t], tA[t])$ is trivial if and only if its image in
                         $\St(A_m[t], tA_m[t])$ is trivial for all maximal ideals $m \trianglelefteq A$.
 \item \label{thm:centr} $K_2^G(A)$ is contained in the centre of $\St(A)$.
\end{thmlist}


\end{thm}
\begin{proof} Follows by the same argument as \cite[Theorem~2.1]{Tu} or \cite[Theorem~2]{S15} \end{proof}



%%%%%%%%%%%%%%%%%%%%%%%%%%%%%%%%%%%%%%%%%%%%%%%%%%%%%%%%%%%%%%%%%%%%%%%%%%%%%%%%%%%%%%%%%%%%
%%%%%%%%%%%%%%%%%%%%%%%%%%%%%%%%%%%%%%%%%%%%%%%%%%%%%%%%%%%%%%%%%%%%%%%%%%%%%%%%%%%%%%%%%%%%
%\section{Quillen-Suslin lgp, Zariski gluing, Nisnevich gluing, $S$-lemma}

\subsection{The automorphisms \texorpdfstring{$\sigma_i$}{\textsigma\textiinferior}} \label{sec:sigma}
Our notation and conventions follows~\cite[\S~4]{VavWE}.
Let $\Phi$ be an irreducible root system with some fixed basis of simple roots $\Pi = \{\alpha_1, \ldots, \alpha_\ell\}$.
We denote by $\Phi^\vee$ the \emph{dual root system of $\Phi$} consisting of vectors $\alpha^\vee = 2\alpha/(\alpha, \alpha)$, $\alpha\in \Phi$.
As usual, $P(\Phi^\vee)$ denotes the lattice spanned by the \emph{fundamental weights $\varpi_i$}.
Recall that $\varpi_i$ are uniquely determined by relations $\langle\varpi_i, \alpha_j^\vee \rangle = (\varpi_i, \alpha_j) = \delta_{ij}.$

Notice that for $\varpi \in P(\Phi^\vee)$ and $\beta \in \ZZ \Phi$ one has $(\varpi, \beta) \in \ZZ$.
Consequently, for $\varepsilon \in R^*$ and $\varpi \in P(\Phi^\vee)$ the identity $\chi_{\varpi, \varepsilon}(\beta) = \varepsilon ^ {(\varpi, \beta)}$
gives a well-defined character $\chi_{\varpi, \varepsilon} \in \Hom(\ZZ \Phi, R^*)$.

Consider the action of $H=\Hom(\ZZ \Phi, R^*)$ on the set of generators $\mathcal{X}_{\Phi, R}$ of the Steinberg group $\St(R)$ defined by
\begin{equation} \chi \cdot x_\alpha(\xi) = x_\alpha(\chi(\alpha) \cdot \xi),\ \chi \in H,\ \alpha\in \Phi,\ \xi \in R. \end{equation}
Since $\chi$ is a character, the above action preserves the set of Steinberg relations $\mathcal{R}_{\Phi, R}$ and,
thus, gives a well-defined action of $H$ on $\St(R)$.

\begin{example} The principal example which motivates the above construction is as follows.
Let $A$ be a ring, take $R = A[t, t^{-1}]$ to be the ring of Laurent polynomials over $A$ and let $\alpha_i \in \Pi$ be some simple root.
Since $t \in R^*$ we can consider the automorphisms $\sigma_i^+$ and $\sigma_i^-$ of $\Stb(\Phi, R)$ given by $\sigma_i^+ = \chi_{\varpi_i, t}$, $\sigma_i^- = \chi_{\varpi_i, t^{-1}}$.
It is easy to see that
\begin{equation}\label{eq:sigma_act} \sigma_i^\pm(x_\alpha(\xi)) = x_\alpha(t^{\pm m_i(\alpha)} \cdot \xi),\end{equation}
where $m_k(\alpha)$ denotes the coefficient in the expansion of $\alpha$ in $\Pi$, i.\,e. $\alpha = \sum m_k(\alpha) \alpha_k$.
\end{example}

One of the key steps of our proof of Suslin lemma for $K_2$ is to define an analogue of $\sigma_i$ for the group $\St(A[t])$.
Of course, we cannot expect such map to be automorphism or even be defined on the whole group $\St(A[t])$.
However, it turns out that for certain $i$ is still possible to define certain subgroups of $\St(A[t])$ and the maps modeling $\sigma_i$ between them.
First, we settle the case $\Phi=\rA_3$ invoking the presentation obtained in \ref{sec:stbA3}.
Then we study the general case using the Curtis-Tits presentation.

\begin{lem} \label{lem:sigma}
 Let $A$ be a local commutative ring, $G = G(\Phi, -)$ where $\Phi$ is an irreducible root system.
 Assume that on the Dynkin diagram of $\Phi$ one can find an endnode numbered $i$ such that $i$ is contained in a subdiagram of type $\rA_3$.

 Then there exists subgroups $N_i^+$, $N_i^-$ of $\St(A[t])$ and homomorphisms $\widetilde{\sigma}_i^+ \colon N_i^+ \to N_i^-$, $\widetilde{\sigma}_i^-\colon N_i^- \to N_i^+$
 compatible with the action of $\sigma_i^\pm$ i.\,e. such that the following diagram commutes.
 \[\begin{tikzcd} N_i^\pm \arrow[r, "\lambda_t^*"] \arrow[d, "\widetilde{\sigma}_i^\pm"]          &  \St(A[t, t^{-1}]) \ar[d, "\sigma_i^\pm"] \\
                  N_i^\mp \arrow[r, "\lambda_t^*"] &  \St(A[t, t^{-1}]) \end{tikzcd}\]

\end{lem}

\begin{proof}[Proof for $\Phi=\rA_\ell$, $\ell\geq 3$ and $i=1$.]
 For $\alpha_i\in\Pi$ denote by $P_i^+$ (resp. $P_i^-$) the subgroup of $\St(A)$ generated by
 $x_\alpha(\xi)$ for $\xi \in A$, $\alpha\in\Sigma_i^+ \cup \Delta_i$ (resp. $\alpha\in\Sigma_i^- \cup \Delta_i$).

 First, we define the subgroups $N_i^\pm$ (this definition also works for $\Phi$ such that $m_i(\widetilde{\alpha})=1$).
 Define $N_{i}^+$ (resp. $N_i^-$) to be the subgroup consisting of $g \in \St(A[t])$ such that $g(0) \in P_i^+$ (resp. $g(0) \in P_i^-$).

 Denote by $j_\ell$ the natural map $\Stb(\Delta_1, A)\to \Stb(\Phi, A)$.
 By the Levi decomposition $P_1^\pm$ is isomorphic to $\UU(\Sigma_1^\pm, A) \rtimes \im(j_\ell)$.

 The stable rank of $A$ equals $1$ hence from the injective stability theorem for $K_2$ (see~\cite[Theorem~4.1]{ST76}) it follows that the map $j_\ell$ is injective for $\ell \geq 3$
 and that $P_1^\pm = \UU(\Sigma_1^\pm, A) \rtimes \Stb(\Delta_1,A)$.
 Clearly, there exists a unique group homomorphism $\sigma_1^\pm \colon P_1^\pm \to \St(\Phi, A[t])$ acting identically on $\Stb(\Delta_1, A)$ and sending
 any generator $x_\alpha(\xi) \in \UU(\Sigma_1^\pm, A)$ (i.\,e. $\alpha\in\Sigma_1^\pm$) to $x_\alpha(t\cdot \xi)$.


 Set $H=\St(A[t], tA[t])$.
 By \cref{item:st-semi} we have $\St(A[t]) = \St(A) \ltimes H$ hence $N_1^\pm \cong P_1^\pm \ltimes H.$
 Define the map $\widetilde{\sigma}^\pm_1 \colon H \to \St(A[t])$ on the generators of $\St(A[t], tA[t])$ as follows ?????.
 %TODO: Fill in the gap using another presentation for relative (!) linear Steinberg groups of rank 3
 A routine check shows that the defining relations ?????? of $H$ are satisfied and that $\sigma_1^\pm$ preserves the action of $P_1^\pm$ on $H$.
 Thus, we obtain a well-defined map $\sigma_1^\pm:N_i^\pm \to \St(A[t])$ fitting into the above commutative diagram.
 The fact that the image of $\sigma_1^\pm$ is contained in $N_i^\mp$ is obvious.
 \renewcommand{\qedsymbol}{} \end{proof}

%%%%%%%%%%%%%%%%%%%%%%%%%%%%%%%%%%%%%%%%%%%%%%%%%%%%%%%%%%%%%%%%%%%%%%%%%%%%%%%%%%%%%%%%%%%%
%%%%%%%%%%%%%%%%%%%%%%%%%%%%%%%%%%%%%%%%%%%%%%%%%%%%%%%%%%%%%%%%%%%%%%%%%%%%%%%%%%%%%%%%%%%%
\section{\texorpdfstring{$\Pro^1$}{P\textonesuperior}-gluing}
Throughout this section $G=G(\Phi, -)$ denotes a simply connected Chevalley--Demazure group scheme of type $\Phi$.

\begin{dfn} \label{def:p1g} Let $F$ be a group-valued functor from $\catname{CRings}$ to $\catname{Groups}$ and let $A$ be a commutative ring.
Consider the following commutative diagram.
\[ \begin{tikzcd} A \ar[r, "i_+"] \ar[d, "i_-"'] & A[t] \ar[d, "j_+"] \\ A[t^{-1}] \ar[r, "j_-"] & A[t, t^{-1}] \end{tikzcd} \]
We say that $F$ satisfies the \emph{$\Pro^1$-glueing property for $A$} if the following sequence of \emph{pointed sets} is exact in the middle term:
\[ \begin{tikzcd} F(A) \ar[r, hookrightarrow, "\Delta_A^F"] & F(A[t]) \times F(A[t^{-1}]) \ar[r, "\pm_A^F"] & F(A[t, t^{-1}]). \end{tikzcd} \]
Here $\Delta^F_A$ denotes the (split injective) diagonal map and, by definition, $\pm_A^F$ maps $(g^+, g^-)$ to $F(j_+)(g^+) \cdot F(j_-)(g^-)^{-1}.$
Notice that a priori $\pm_A^F$ is only a morphism of pointed sets but if $F$ takes values in abelian groups then $\pm_A^F$ is also a morphism of groups.

An equivalent way to formulate $\Pro^1$-glueing property is as follows:
$F(j_+)$ and $F(j_-)$ are injective and the intersection of their images coincides with the image of $F(j_+ i_+) = F(j_- i_-).$ \end{dfn}

The main result of this section is the following theorem which generalizes \cite[Theorem~5.1]{Tu} to Chevalley groups.
Notice that a $K_1$-analogue of the result below has been established in a much greater generality by the second-named author (see~\cite[Theorem~1.1]{St-poly}).
%TODO: TLP is not the only ingredient needed in the proof
\begin{thm} \label{thm:p1} Assume that $G$ satisfies Tulenbaev lifting property~\ref{def:tlp}.
Then the Steinberg group functor $\St(-)$ satisfies $\Pro^1$-glueing property for an arbitrary commutative ring $A$. \end{thm}
\begin{proof}
 Let $(g^+, g^-)$ be an element of $\St(A[t]) \times \St(A[t^{-1}])$ such that the equality $g^+ = g^-$ holds in $\St(A[t, t^{-1}])$.

 Let $m$ be a maximal ideal of $A$.
 By \cref{prop:p1g} below the functor $\St(-)$ satisfies $\Pro^1$-glueing property for the local ring $A_m$ hence
 $(\lambda_m^*(g_+), \lambda_m^*(g_-)) = \Delta_{A_m}(\lambda_m^*(g^+)(0))$ and in the groups $\St(A_m[t])$ and $\St(A_m[t^{-1}])$ we have the equalities:
 $$\lambda_m^*(g^+ \cdot g^+(0)^{-1}) = \lambda_m^*(g^+) \cdot {\lambda_m^*(g^+)(0)}^{-1} = 1; \qquad \lambda_m^*(g^-\cdot g^+(0)^{-1}) =1. $$
 Now, by the local-global principle for $\St(-)$ (see \cref{thm:lg-k2}) these equalities hold globally and $(g^+, g^-) = \Delta_A(g^+(0))$, as claimed.
\end{proof}

\begin{rem} \label{rem:stk2} It is clear that if the functor $\St$ satisfies $\Pro^1$-glueing property for $A$ then so does the functor $K_2^G$.
 The converse statement also holds, indeed, if $(g^+, g^-)\in\Bigker(\pm^{\Stb}_A)$ then inside $E^G(A[t, t^{-1}])$ we have the equality:
 $$\varphi(\St(j_+)(g^+)) = \varphi(\St(j_-)(g^-)) \in E^G(A[t]) \cap E^G(A[t^{-1}]) = E^G(A).$$
 Consequently, we can find $g_0 \in \St(A)$ so that $(g^+ g_0^{-1}, g^- g_0^{-1}) \in \Bigker(\pm_A^{K_2})$ and
 it remains to apply the $\Pro^1$-glueing property for $K_2^G$. \end{rem}

\begin{cor} Let $A$ be any commutative ring and $f\in A[t]$ be a monic polynomial.
Then the map $K_2^G(A[t])\to K_2^G(A[t]_f)$ is injective. \end{cor}
\begin{proof}
%TODO: Write proof
\end{proof}

%%%%%%%%%%%%%%%%%%%%%%%%%%%%%%%%%%%%%%%%%%%%%%%%%%%%%%%%%%%%%%%%%%%%%%%%%%%%%

\subsection{The field case}

%%%%%%%%%%%%%%%%%%%%%%%%%%%%%%%%%%%%%%%%%%%%%%%%%%%%%%%%%%%%%%%%%%%%%%%%%%%%%



Throughout this section $k$ denotes an arbitrary field.
\begin{thm} \label{thm:k[t]}
Assume that $G=G(\Phi, -)$ and $\Phi$ is any irreducible root system  of rank $\geq 2$.
\begin{thmlist}
\item \label{satz1} The subgroup $K_2(\Phi,k[t]) \trianglelefteq \St(\Phi,k[t])$
is generated by symbols $\{u,v\}_\alpha$, $u,v\in k^*,\alpha\in\Phi$.
\item As a consequence, the canonical injection $K_2(\Phi,k) \hookrightarrow K_2(\Phi,k[t])$ is an isomorphism,
and $K_2(k[t])=K_2(k)$ is central in $\Stb(\Phi,k[t])$.
\end{thmlist} \end{thm}
\begin{proof}
See \cite[Satz~1]{Re75} and the corollary after it.
\end{proof}

\begin{cor}\label{cor:k[t]inj}
Let $G$ be as in Theorem~\ref{thm:k[t]}. Then the functors $\St$, $K_2^G$ satisfy $\Pro^1$-glueing property for $k$. \end{cor}
\begin{proof} By \cref{rem:stk2} it suffices to prove the assertion only for the functor $K_2^G$.
By the previous theorem $K_2^G(i_+)$ and $K_2^G(i_-)$ are isomorphisms hence the morphisms $K_2^G(j_+)$ and $K_2^G(j_-)$ are split injective
and $\im(j_+i_+) = \im (j_-i_-) = \im(j_+) = \im(j_-)$.
\end{proof}

\begin{thm}\label{thm:k[t+-1]}
Let $\Phi$ be an irreducible root system of rank $\geq 2$, $\Phi\neq G_2$.
Then for any long root $\alpha\in\Phi$ one has
$$
K_2(\Phi,k[t^{\pm 1}])=K_2(\Phi,k)\oplus
\left<\{t,u\}_\alpha,\ u\in k^*\right>.
$$
In particular, $K_2(\Phi,k[t^{\pm 1}])$ is central in $\Stb(\Phi,k[t^{\pm 1}])$.
Here $K_2(\Phi,k)$ is considered as a subgroup of $K_2(\Phi,k[t^{\pm 1}])$ via the natural injection.
\end{thm}
\begin{proof}
Set $H=\left<\{t,u\}_\alpha,\ u\in k^*\right>$.
By~\cite[Korollar 4]{Hur77} the group $K_2(\Phi,k[t^{\pm 1}])$ is generated by $K_2(\Phi,k)$ and
$H$. By~\cite[Prop. 1.1 (S1)]{Ste73} one has $\{1,u\}_\alpha=1$
for any $\alpha\in\Phi$. Hence $H$ is in the kernel of the natural projection $K_2(\Phi,k[t^{\pm 1}])\to K_2(\Phi,k)$
sending $t$ to 1, and $H\cap K_2(\Phi,k)=1$. By~\cite[Prop. 1.3 (a)]{Ste73} symbols are central in
$\Stb(\Phi,k[t^{\pm 1}])$, hence $K_2(\Phi,k[t^{\pm 1}])=K_2(\Phi,k)\oplus H$.
\end{proof}

%%%%%%%%%%%%%%%%%%%%%%%%%%%%%%%%%%%%%%%%%%%%%%%%%%%%%%%%%%%%%%%%%%%%%%%%%%%%%

\subsection{Tulenbaev's section 3}

%%%%%%%%%%%%%%%%%%%%%%%%%%%%%%%%%%%%%%%%%%%%%%%%%%%%%%%%%%%%%%%%%%%%%%%%%%%%%

For the rest of this section $A$ denotes an arbitrary commutative local ring with the maximal ideal $m$
and the residue field $k$.
We denote by $\pi$ the canonical projection $A \rightarrow k$.
Throughout this section we will employ the following notation:
\begin{itemize}
 \item $R$ denotes the Laurent polynomial ring $A[t, t^{-1}]$;
 \item $B$ denotes the subring $A[t] + m[t^{-1}]$ of $R$ consisting of Laurent polynomials $f(t,t^{-1})$ whose coefficients of terms of negative degree belong to $m$;
 \item $I$ denotes the ideal $m[t, t^{-1}]$ of $R$ (which can be also considered as an ideal of $B$).
\end{itemize}

Note that since $A$ is local, then $K_2(\Phi,A)=\Stsym(\Phi,A)$
by~\cite[Theorem 2.13]{Ste73}. In particular, $K_2(\Phi,A)$ is central in $\Stb(\Phi,A)$,
and $W(\Phi,A)/H(\Phi,A)$ is naturally isomorphic
to the Weyl group $W(\Phi)$.

Our first result is analogous to \cite[Lemma~3.1(e)]{Tu} (cf. also with \cite[\S~2.3A]{HOM}).
\begin{lem}\label{lem:bruhat}
Let $\Phi$ be any irreducible root system.
Let $\Phi^+$, $\Phi^{+'}$ be two systems of positive roots in $\Phi$.
 \begin{lemlist}
\item  The Steinberg group
$\Stb(\Phi,A)$ admits the following analogue of the Bruhat decomposition:
\begin{equation}\label{eq:bruhat}
\Stb(\Phi,A) =\bigsqcup_{w\in W(\Phi)} \left(\UU(\Phi^{+'}, A)\cdot wH(\Phi,A)\cdot\UU(\Phi^+, A) \cdot
\ker\bigl(\Stb(\Phi,A) \xrightarrow{\pi^*} \Stb(\Phi,k)\bigr)\right).
\end{equation}
\item Assume that $uwhvl=u'w'h'v'l'$ for some $u,u'\in \UU(\Phi^{+'}, A)$, $w,w'\in W(\Phi)$,
$h,h'\in H(\Phi,A)$, and $l,l'\in \ker\bigl(\Stb(\Phi,A) \xrightarrow{\pi^*} \Stb(\Phi,k)\bigr)$. Then
$w=w'$,
$$
h^{-1}h'\in \Stsym(\Phi,A)\cdot\ker\bigl(H(\Phi,A) \xrightarrow{\pi^*} H(\Phi,k)\bigr),
$$
and
there exist $a\in \UU(\Phi^{+'},m)$ such that $w^{-1}(u^{-1}u'a)w\in \UU(\Phi^+,R)$,
and $b\in\UU(\Phi^+,m)$ such that
$$
b=v^{-1}(u^{-1}u')^{wh}v'=l(l')^{-1}.
$$
 \end{lemlist}
\end{lem}
\begin{proof}
For any field $k$, the group $G(\Phi,k)=E(\Phi,k)$ admits Bruhat decomposition, hence
$K_2(\Phi,k)\le H(\Phi,k)$ implies
$$
\Stb(\Phi,k)=\bigsqcup_{w\in\WW(\Phi)} \UU(\Phi^{+'}, k) wH(\Phi,k)\UU(\Phi^+, k).
$$
Let $w_0\in W(\Phi)$ be such that $v_0(\Phi^{+'})=\Phi^+$. Then also
$$
\Stb(\Phi,k)=w_0^{-1}H(\Phi,k)\Stb(\Phi,k)=\bigsqcup_{w\in\WW(\Phi)} \UU(\Phi^{+'}, k) wH(\Phi,k)\UU(\Phi^+, k).
$$
As a consequence, the first three factors in the right hand side of the decomposition~\eqref{eq:bruhat} are mapped
epimorphically onto $\Stb(\Phi,k)$ and the last factor coincides with $\Bigker(\pi^*)$
 from which the first assertion of the lemma follows. The second assertion follows from the unicity
of the Bruhat decomposition in $G(\Phi,k)$.
\end{proof}

\begin{lem}\label{lem:tul3.1zh}
Denote by $\Stsymt(\Phi,A,m)$ the subgroup of $\Stsym(\Phi,A[t^{\pm 1}])$ generated by
all symbols of the form $\{t,u\}_\alpha$, $\alpha\in\Phi$, $u\in 1+m$. Then there is an injective
homomorphism $\phi:\Stsymt(\Phi,A,m)\to\Stb(\Phi,A[t^{\pm 1}],mA[t^{\pm 1}])$, natural in $(A,m)$,
such that the composition of the canonical map $\Stb(\Phi,A[t^{\pm 1}],mA[t^{\pm 1}])\to
\Stb(\Phi,A[t^{\pm 1}])$ with $\phi$ equals $\id_{\Stsymt(\Phi,A,m)}$.
\end{lem}
\begin{proof}
The map $\phi$ can be defined as follows:
\begin{multline*}
\phi(\{t,u\}_\alpha)=y_\alpha(t(u-1))y_{-\alpha}(-t^{-1}(u^{-1}-1))^{x_\alpha(-t)}\cdot \\
\cdot y_\alpha(t(u-1))^{w_\alpha(-t)}
y_\alpha(u-1)y_{-\alpha}(-(u^{-1}-1))^{x_\alpha(-1)}y_\alpha(u-1)^{w_\alpha(-1)}.
\end{multline*}
If we assume that $\Phi$ is non-symplectic and we restrict ourselves to just one root $\alpha$,
then it is enough to check that $\{t,u\}_\alpha\neq 1$ in $\Stb(\Phi,A[t^{\pm 1}])$ for any $u\neq 1$.
If $A$ is an integral domain (e.g. $A$ regular), this follows from the injectivity of $A^*\to K^*$, where $K$ the
fraction field of $A$, together with the injectivity of $K^*\to \Stb(A_2,K(t))$ (see e.g.~\cite[Remark 1
on p. 208]{DeSte-dvr}) and the
stability~\cite[Theorem A.2]{DeSte-dvr} applied to $A_2\to\Phi$ and the field $K(t)$.

....to finish....
\end{proof}

\begin{lem}\label{lem:sigma-X}
Let $\Phi$ be an irreducible root system of rank $l\ge 3$ and of type $A_l$ ($l\ge 3$), $C_l$ ($l\ge 3$), $D_l$ ($l\ge 4$),
$E_6$ or $E_7$. Let $\alpha_i\in\Pi$, $1\le i\le l$, be a simple root of $\Phi$ such that the
parabolic subgroup $P_i$ of $G(\Phi,-)$ has abelian unipotent radical. Denote by $\Stb P_i^-(\Phi,A)$
the subgroup of $\Stb(\Phi,A)$ generated by $x_\alpha(u)$, $\alpha\in\Delta_i\cup(-\Sigma_i)$, $u\in A$,
and by $H(\Phi,A)$ (?).
Then there exists a group homomorphism
$$
\delta_i:\Stb(\Phi,A[t],tA[t])\to \Stb P_i^-(\Phi,A)\cdot \Stb(\Phi,A[t],tA[t])
$$
such that the following diagram commutes:
 \[\begin{tikzcd}
\Stb(\Phi,A[t],tA[t]) \arrow[r, "j_+"] \arrow[d, "\delta_i"] & \Stb(\Phi,A[t, t^{-1}])  \ar[d, "\chi_{\varpi_i,t}"] \\
\Stb P_i^-(\Phi,A)\cdot \Stb(\Phi,A[t],tA[t]) \arrow[r, "j_+"] &  \Stb(\Phi,A[t, t^{-1}])
\end{tikzcd}\]
\end{lem}
\begin{proof}
The cases $A_l$ and $C_l$ are done using another presentation. Other cases are done by amalgamation
of $A_3$-pieces....
\end{proof}

\begin{lem}\label{lem:parab-pairs}
Let $\Phi$ be an irreducible root system of rank $l\ge 3$ and of type $A_l$ ($l\ge 3$), $C_l$ ($l\ge 3$),
$D_l$ ($l\ge 4$), $E_6$, $E_7$ or $E_8$.
Let $\alpha_i\in\Pi$ be the simple root adjacent to $\alpha_l$ in the Dynkin diagram of $\Phi$
(note that $P_l$ has abelian unipotent radical if $\Phi\neq E_8$, and extraspecial if $\Phi=E_8$).
Set $\Phi^{+'}=w_{\alpha_l}(\Phi^+)$.
Then $\Phi^{+'}\setminus\Phi^+=-\alpha_l$, $\Phi^+\setminus \Phi^{+'}=\alpha_l$, and
$\Sigma_i\setminus\Sigma_l\subseteq w_{\alpha_l}(\Sigma_l)$.
\end{lem}
\begin{proof}
The first two claims are obvious. The last claim follows from the fact that for any root
$\alpha\in\Sigma_i\setminus\Sigma_l$ one has $w_{\alpha_l}(\alpha)=\alpha+\alpha_l$.
\end{proof}

\begin{lem}\label{lem:tul3.3}
 Let $\Phi$ be an irreducible root system of rank $\ge 2$, let $\Pi$ be a system of simple roots in $\Phi$,
and let $J\subseteq\Pi$ be such that $|J|\ge 2$.
%Set
%$$
%\alpha_J=\sum\limits_{\beta\in J}m_\beta(\alpha)\beta\ \mbox{for any}\ \alpha\in\Phi,
%$$
%and
%$$
%\Sigma_J=\{\alpha\in\Phi\ |\ m_\beta(\alpha)>0\ \mbox{for at least one}\ \beta\in J\}.
%$$
Let $S$ be any group, and $n\in\mathbb{N}$. For any $b\in B$ let $|b|\in\ZZ$ be such that $t^{|b|}$ is the
smallest power of $t$ occuring in $b$.
Assume that there are elements $s_\alpha(u)\in S$ for all $\alpha\in\Sigma_J\cup(-\Sigma_J)$ and
$u\in B$ such that $|u|\ge -n$, satisfying the relations
\begin{enumerate}
\item $s_\alpha(u)s_\alpha(v)=s_\alpha(u+v)$ for all $\alpha\in \Sigma_J\cup(-\Sigma_J)$, $u,v\in B$,
$|u|,|v|\ge -n$;
\item $[s_\alpha(u),s_\beta(v)]=\prod\limits_{i,j\in\mathbb{N}}s_{i\alpha+j\beta}(N_{\alpha\beta ij}u^iv^j)$
for all $\alpha,\beta\in \Sigma_J\cup(-\Sigma_J)$ such that $i\alpha_J\neq -j\beta_J$ for all $i,j\in\mathbb{N}$,
and all $u,v\in B$ such that $i|u|+j|v|\ge -n$ whenever $N_{\alpha\beta ij}\neq 0$.
\end{enumerate}
If $n=1$, then the map $x_\alpha(u)\mapsto s_\alpha(u)$ extends to a group homomorphism $\Stb(\Phi,B)\to S$.
\end{lem}
\begin{proof}

\end{proof}
%%%%%%%%%%%%%%%%%%%%%%%%%%%%%%%%%%%%%%%%%%%%%%%%%%%%%%%%%%%%%%%%%%%%%%%%%%%%%

\subsection{Tulenbaev's section 4}

%%%%%%%%%%%%%%%%%%%%%%%%%%%%%%%%%%%%%%%%%%%%%%%%%%%%%%%%%%%%%%%%%%%%%%%%%%%%%

\begin{thm}\label{thm:tul4.1}
Let $\Phi$ be an irreducible root system of rank $l\ge 3$ of type $A_l$, $C_l$, $D_l$ ($l\ge 4$), $E_6$ or $E_7$.
Then $\ker\bigl(\Stb(\Phi,A[t],m[t])\to\Stb(\Phi,A[t])\bigr)$ surjects onto
$\ker\bigl(\Stb(\Phi,A[t^{\pm 1}],m[t^{\pm 1}])\to\Stb(\Phi,A[t^{\pm 1}])\bigr)$.
\end{thm}
\begin{proof}
Set
$$
\tilde B=H(\Phi,A[t^{\pm 1}])\cdot\UU(\Phi^+,A[t^{\pm 1}])\le\Stb(\Phi,A[t^{\pm 1}]),
$$
and
$$
\tilde D=\phi(\Stsymt(\Phi,A,m))\le \Stb(\Phi,A[t^{\pm 1}],m[t^{\pm 1}])
$$
in the notation of Lemma~\ref{lem:tul3.1zh}. Consider the set ofequivalence classes
$$
V=\Stb(\Phi,A[t])\times \tilde B\times \Stb(\Phi,A[t^{\pm 1}],mA[t^{\pm 1}])/\sim,
$$
where $( a,b,\beta)\sim( a',b',\beta')$ if and only if there is $\gamma\in\Stb(\Phi,A[t],mA[t])$,
$p\in\UU(\Phi^+,A[t])$ and $\mu\in \tilde D\cdot\UU(\Phi^+,mA[t^{\pm 1}])$ such that
$$
\tilde a= a\gamma_1^{-1}t^{-1},\quad \tilde b=pt\mu,\quad \tilde\beta=\mu^{-1}(\gamma_2)b\beta,
$$
where $\gamma_1$ and $\gamma_2$ are the images of $\gamma$ in the respective groups. The elements of $V$
will be denoted $[ a,b,\beta]$.

{\bf We will define a map $\sigma:V\to V$.} By Lemma~\ref{lem:bruhat} and the definition of $V$ every element of $V$
can be written in the form $[ a w,b,\beta]$ for some $ a\in\Stb(\Phi,A[t],tA[t])\cdot\UU(\Phi^+,A)$,
$w\in W(\Phi,A)\le\Stb(\Phi,A)$, $b\in\tilde B$ and $\beta\in \Stb(\Phi,A[t^{\pm 1}],mA[t^{\pm 1}])$.
We define
$$
\sigma|_{\Stb(\Phi,A[t],tA[t])}=\delta_l:\Stb(\Phi,A[t],tA[t])\to\Stb(\Phi,A[t],tA[t])\Stb P_l^-(\Phi,A)\le[\Stb(\Phi,A[t]),1,1],
$$
where $\delta_l$ the homomorphism constructed in Lemma~\ref{lem:sigma-X}.
For any $\alpha\in\Phi$, $u\in A[t^{\pm 1}]$ we have a homomorphism
$$
\chi_{\varpi_l,t}:X_\alpha(A[t^{\pm 1}])\to X_\alpha(A[t^{\pm 1}]),\quad
x_\alpha(u)\mapsto x_\alpha(t^{m_{\alpha_l}(\alpha)}u).
$$
In particular, this induces a homomorphism
$\sigma_U=\chi_{\varpi_l,t}|_{\UU(\Phi^+,A)}:\UU(\Phi^+,A)\to \UU(\Phi^+,A[t])$.
Combining $\delta_l$ and $\sigma_U$, we obtain a homomorphism (check!!!)
$$
\sigma:\Stb(\Phi,A[t],tA[t])\cdot\UU(\Phi^+,A)\to \Stb(\Phi,A[t]).
$$
Now we define
$$
\sigma\cdot[ a w,b,\beta]=[\sigma( a)w,w^{-1}\chi_{\varpi_l,t}(wb),\chi_{\varpi_l,t}(\beta)].
$$
Note that, clearly, $w^{-1}\chi_{\varpi_l,t}(w)\in H(\Phi,A[t^{\pm 1}])$.

{\bf Next we define a map $\sigma':V\to V$.} Set $\Phi^{+'}=w_{\alpha_l}(\Phi^+)$.
By Lemma~\ref{lem:bruhat} and the definition of $V$ every element of $V$
can be written in the form $[ a w,b,\beta]$ for some $ a\in\Stb(\Phi,A[t],tA[t])\cdot\UU(\Phi^{+'},A)$,
$w\in W(\Phi,A)\le\Stb(\Phi,A)$, $b\in\tilde B$ and $\beta\in \Stb(\Phi,A[t^{\pm 1}],mA[t^{\pm 1}])$.
Note that $w_{\alpha_l}(\Pi)$ is a set of simple roots of $\Phi$ contained in $\Phi^{+'}$, with
$w_{\alpha_l}(\alpha_l)=-\alpha_l$ playing the role of $\alpha_l$. Then by Lemma~\ref{lem:sigma-X}
there is a homomorphism
$$
\delta_l':\Stb(\Phi,A[t],tA[t])\to\Stb(\Phi,A[t],tA[t])\Stb {P'_l}^-(\Phi,A),
$$
compatible with $\chi_{w_{\alpha_l}(\varpi_l),t}$, where $\Stb {P'_l}^-(\Phi,A)$ denotes the subgroup
of $\Stb(\Phi,A)$ corresponding to the parabolic set of roots $w_{\alpha_l}(\Delta_l\cup\Sigma_l)$.
We define
$$
\sigma'|_{\Stb(\Phi,A[t],tA[t])}=\delta'_l:\Stb(\Phi,A[t],tA[t])\to\Stb(\Phi,A[t],tA[t])\Stb {P'_l}^-(\Phi,A)\le[\Stb(\Phi,A[t]),1,1].
$$
Similarly, for any $\alpha\in\Phi$, $u\in A[t^{\pm 1}]$ we consider a homomorphism
$$
\chi_{w_{\alpha_l}(\varpi_l),t}:X_\alpha(A[t^{\pm 1}])\to X_\alpha(A[t^{\pm 1}]),\quad
x_\alpha(u)\mapsto x_\alpha(t^{m'_{-\alpha_l}(\alpha)}u),
$$
where $m'_{w_{\alpha_l}(\alpha_i)}(\alpha)$, $1\le i\le l$, is the coefficient of $w_{\alpha_l}(\alpha_i)$
in the decomposition of $\alpha\in\Phi$ with respect to $w_{\alpha_l}(\Pi)$.
We also consider a homomorphism
$\sigma'_U=\chi_{w_{\alpha_l}(\varpi_l),t}|_{\UU(\Phi^{+'},A)}:\UU(\Phi^{+'},A)\to \UU(\Phi^{+'},A[t])$.
Combining $\delta'_l$ and $\sigma'_U$, we obtain a homomorphism
$$
\sigma':\Stb(\Phi,A[t],tA[t])\cdot\UU(\Phi^{+'},A)\to \Stb(\Phi,A[t]).
$$
Now we define
$$
\sigma'\cdot[ a w,b,\beta]=[\sigma'( a)w,w^{-1}\chi_{w_{\alpha_l}(\varpi_l),t}(wb),
\chi_{w_{\alpha_l}(\varpi_l),t}(\beta)].
$$
As in the case of $\sigma$, we have $w^{-1}\chi_{w_{\alpha_l}(\varpi_l),t}(w)\in H(\Phi,A[t^{\pm 1}])$.

Now we show that $\sigma,\sigma':V\to V$ are correctly defined and bijective. We prove that for $\sigma'$,
the case of $\sigma$ being analogous.....
%Assume that
%$[ a x,b,\beta]=[\tilde a\tilde w,\tilde b,\tilde\beta]$ for some $\tilde a\in

{\bf Next we show that $\sigma$ and $\sigma'$ commute.}
First we note that if $\sigma\cdot[a,1,1]=[x,y,1]$, then for any $b\in\tilde B$ and
$\beta\in\Stb(\Phi,A[t^{\pm 1}],mA[t^{\pm 1}])$ one has
$$
\sigma\cdot[a,b,\beta]=[x,y\chi_{\varpi_l,t}(b),\chi_{\varpi_l,t}(\beta)],
$$
and a similar equality holds for $\sigma'$. Since
$\chi_{\varpi_l,t}$ and $\chi_{w_{\alpha_l}(\varpi_l),t}$ commute on $\Stb(\Phi,A[t^{\pm 1}])$
and on $\Stb(\Phi,A[t^{\pm 1}],mA[t^{\pm 1}])$, it is enough to check that $\sigma$ and $\sigma'$
commute on any element of the form $[a w,1,1]\in V$, where $a\in\Stb(\Phi,A[t],tA[t])\cdot\UU(\Phi^+,A)$,
$w\in W(\Phi,A)$. We can write $a=a_0\cdot x_{\alpha_l}(u)$, where
$a_0\in \Stb(\Phi,A[t],tA[t])\cdot\UU(\Phi^+\cap\Phi^{+'},A)$.....

%Consider two cases.
%First, if $u\in m$ or $w^{-1}(\alpha_l)\in\Phi^+$, we can transfer
%the factor $x_{\alpha_l}(u)$ into $\beta$ or $b$ respectively, and thus assume that $u=0$. In this
%case one readily sees that $\sigma$ and $\sigma'$ commute on $[a_0,b,\beta]$, since
%$\chi_{\varpi_l,t}$ and $\chi_{w_{\alpha_l}(\varpi_l),t}$ commute (check!!!).
%Second, assume that $u\in A\setminus m=A^*$ and $w^{-1}(\alpha_l)\in \Phi^-$. One has
%\begin{multline}
%\sigma'\sigma\cdot[a_0x_{\alpha_l}(u)w,1,1]=\\
%=\sigma'\cdot
%[\sigma(a_0)x_{\alpha_l}(tu)w,w^{-1}\chi_{\varpi_l,t}(w),1]=\\
%=[\sigma'\bigl(\sigma(a_0)\cdot \sigma(a_0)(0)^{-1}x_{\alpha_l}(tu)^{\sigma(a_0)(0)^{-1}}\bigr)\cdot \sigma'\bigl(\sigma(a_0)(0)\bigr)
%\end{multline}
%On the other hand, one has
%\begin{multline}
%\sigma\sigma'\cdot[a_0x_{\alpha_l}(u)w,1,1]=\\
%\sigma\sigma'\cdot[a_0x_{-\alpha_l}(u^{-1})w_{-\alpha_l}(-u^{-1})w,x_{-\alpha_l}(u^{-1})^w ,1]=\\
%=\sigma\cdot[\sigma'(a_0)x_{-\alpha_l}(tu^{-1})w_{-\alpha_l}(-u^{-1})w,w^{-1}w_{-\alpha_l}(-u^{-1})^{-1}
%\chi_{w_{\alpha_l}(\varpi_l),t}\bigl(w_{-\alpha_l}(-u^{-1})w\bigr),1]
%\end{multline}

{\bf Now we define the action of $\Stb(\Phi,A[t^{\pm 1}]))$ on $V$.} For any $\alpha\in\Phi$ and
$u\in A[t]\subseteq A[t^{\pm 1}]$
we set
$$
x_\alpha(u)\cdot [a,b,\beta]=[x_\alpha(u)a,b,\beta].
$$
Next we want to use Lemma~\ref{lem:tul3.3} applied to the set $J=\{\alpha_i,\alpha_l\}$, where $\alpha_i$
is the simple root adjacent to $\alpha_l$. For any $\alpha\in\pm\Sigma_l$ and any $u\in A$ we set
$$
x_\alpha(t^{-1}u)\cdot [a,b,\beta]=\sigma^{\mp 1}\cdot x_\alpha(u)\cdot\sigma^{\pm 1} [a,b,\beta].
$$
For any $\alpha\in\pm\Sigma_i$ and any $u\in A$ we set
$$
x_\alpha(t^{-1}u)\cdot [a,b,\beta]={\sigma'}^{\mp 1}\cdot x_\alpha(u)\cdot{\sigma'}^{\pm 1} [a,b,\beta]
$$
(cf. Lemma~\ref{lem:parab-pairs}). Then we check that this action satisfies the properties required
in Lemma~\ref{lem:tul3.3}. This defines an action of $\Stb(\Phi,A[t^{\pm 1}]))$ on $V$.

The last step is to check that for any $\beta\in \Stb(\Phi,A[t^{\pm 1}],mA[t^{\pm 1}])$ the action of
its image in $\Stb(\Phi,A[t^{\pm 1}])$ on $[1,1,1]$ gives $[1,1,\beta]$.

\end{proof}


%%%%%%%%%%%%%%%%%%%%%%%%%%%%%%%%%%%%%%%%%%%%%%%%%%%%%%%%%%%%%%%%%%%%%%%%%%%%%

\subsection{Suslin's lemma}

%%%%%%%%%%%%%%%%%%%%%%%%%%%%%%%%%%%%%%%%%%%%%%%%%%%%%%%%%%%%%%%%%%%%%%%%%%%%%


\begin{lem} \label{lem:tulinj}
Assume that $G$ satisfies Tulenbaev lifting property~\ref{def:tlp}.
Then the map $i$ in the following commutative diagram of groups is injective.
\begin{equation} \label{diag:cs} \begin{tikzcd}
C_{B} \ar[r, hookrightarrow] \ar[d, "k", twoheadrightarrow] & \St(B, I) \ar[r, "\mu_B" near start] \ar[d, "j"] & \ar[d, "i"] \St(B) \ar[r, "\pi_B", twoheadrightarrow] & \St(k[t]) \ar[d, hookrightarrow] \\
C_{R} \ar[r, hookrightarrow]                                & \St(R, I) \ar[ur, dashrightarrow, "\varphi"] \ar[r, "\mu_R"' near start] & \St(R) \ar[r, "\pi_R"', twoheadrightarrow] & \St(k[t, t^{-1}]) \\ \end{tikzcd} \end{equation} \end{lem}
\begin{proof} First of all, notice that by \cref{cor:k[t]inj} the vertical map in the right-hand side of the diagram is injective.
 Invoking Tulenbaev's property~\ref{def:tlp} we also find a lifting map $\varphi$ in the central square of the diagram.

 Let $g \in \St(B)$ be an element of $\Bigker(i)$.
 Since $g$ also lies in $\Bigker(\pi_B)$ it comes from some $\widetilde{g} \in \St(B, I)$ via $\mu_B$.
 But $j(\widetilde{g})$ lies in $C_R$, hence, by \cref{prop:kersurj} below it comes from some $\widehat{g} \in C_B$ via $k$.
 Finally, $g = \varphi(j(\widetilde{g})) = \varphi(k(\widehat{g})) = \mu_B(\widehat{g}) = 1,$ as claimed. \end{proof}

The following result is analogous to \cite[Proposition~4.1]{Tu}.

\begin{lem} The map $j$ in the diagram \eqref{diag:cs} is surjective for any commutative ring $A$. \end{lem}
%The argument below is taken from A. Stavrova's 08/11/15 letter
\begin{proof}
 Let $\alpha_i$ be arbitrary simple root from $\Pi$ and let $\Sigma_i$ (resp. $\Sigma_i^-$) denote the special subsets of roots
 consisting of $\alpha \in \Phi$ such that $m_i(\alpha) > 0$ (resp. $m_i(\alpha) < 0$).

 By \cref{lem:Zgen} the sets $\mathcal{Z}(\Sigma_i, R, I)$ and $\mathcal{Z}(\Sigma_i^-, R, I)$ both generate $\St(R, I)$.
 Similarly, the subsets $\mathcal{Z}(\Sigma_i, B, I) \subseteq \mathcal{Z}(\Sigma_i, R, I)$,
                        $\mathcal{Z}(\Sigma_i^-, B, I) \subseteq \mathcal{Z}(\Sigma_I, R, I)$ both generate
                        the image of $j\colon \St(B, I) \to \St(R, I)$.

 %TODO: Define this action in a more explicit way.
 Using formula~\eqref{eq:sigma_act} we can calculate how the powers of the automorphism $\sigma_i$ act on these generating sets.
 Indeed, for $z_\alpha(s, \xi) \in \mathcal{Z}(\Sigma_i^-, R, I)$ we have
 \[ \sigma_i^N (z_\alpha(s, \xi)) = \sigma_i^N (x_\alpha(0, s)^{x_{-\alpha}(\Delta(\xi))}) =
     z_\alpha(t^{N m_i(\alpha)} \cdot s, t^{-N m_i(\alpha)} \cdot \xi). \]
 It is clear that for sufficiently large $N$ the element $t^{-N m_i(\alpha)} \cdot \xi$ belongs to $A[t] \subseteq B$.
 Consequently, for arbitrary $g \in \St(R, I)$ there exists $N > 0$ such that $\sigma^N(g)$ lies in the image of $j$.
 On the other hand, by a similar calculation we get for $z=z_{\alpha}(s, \xi) \in \mathcal{Z}(\Sigma_i, B, I)$
 that the element $\sigma^{-N}(z)$ ($N > 0$) still lies in $\mathcal{Z}(\Sigma_i, B, I)$.
 Clearly, this implies that the image of $j$ is preserved by $\sigma_i^{-1}$.
 The assertion of the lemma now follows from these two statements:
 \[ g = \sigma^{-N} \sigma^N (g) \in \sigma^{-N}(\im(j)) \subseteq \im(j). \qedhere\] \end{proof}

\begin{prop} \label{prop:kersurj} Under the assumptions of \cref{lem:tulinj} the map $k$ in the diagram \eqref{diag:cs} is surjective. \end{prop}
\begin{proof}[Sketch/draft of the proof]
The argument presented below only may work under additional assumption that $m_i(\widetilde{\alpha}) = 1$ ($i$ is as in the statement of \cref{lem:sigma}).

Consider the following set
$$X = \St(B) \times \widetilde{B}(R) \times \St(R, I) /  \simeq.$$
The congruence relation we impose should be similar to that used in Tulenbaev's paper.
In particular, for $h \in \St(B, I)$ we should have the following relation in $X$
$$[\mu_B(h), 1, g] \simeq [1, 1, j(h) g].$$
Notice that instead of $\St(A[t])$ (used by Tulenbayev in the first factor) we use $\St(B)$.

TODO: We should prove that there is a well-defined action of $\St(R)$ on $X$.

Now let $g$ be an element of $C_R$. By the previous lemma we have for some $g' \in \St(B, I)$
\[ [1, 1, g] = [1, 1, j(g')] = [\mu_B(g'), 1, 1] = i\mu_B(g')[1, 1, 1] = [1, 1, 1]. \qedhere \]
\end{proof}


\subsection{Proof of the main result}

The following result is analogous to \cite[Proposition~4.3]{Tu}.
It plays the same role in our proof of $\Pro^1$-glueing for $K_2$ as generalized Suslin lemma (cf. \cite[Theorem~2.16]{Abe}) does in the corresponding proof for $K_1$.
\begin{prop} \label{prop:p1g} The functors $\St$, $K_2^G$ satisfy $\Pro^1$-glueing property for arbitrary local ring $A$.
\end{prop}
\begin{proof}
%TODO: Write proof
\end{proof}

\DeclareRobustCommand{\VAN}[2]{#2}
\printbibliography

\end{document}
