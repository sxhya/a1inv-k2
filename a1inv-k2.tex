\documentclass[oneside,12pt]{amsart}
\usepackage{amssymb, amsxtra, amsmath, amstext, amsthm, amsfonts, amscd, enumitem, thmtools,
            tikz, graphicx, tikz-cd, ifthen}
\usepackage[breaklinks=true, pdfencoding=auto]{hyperref}            
\usepackage[T2A]{fontenc}
\usepackage[cp1251]{inputenc}
\usepackage[all]{xy}
\usepackage[notref,notcite]{showkeys}
\usepackage[capitalize]{cleveref}
\usepackage[toc,page]{appendix}
\usepackage[english]{babel}
\usetikzlibrary{matrix,arrows,cd}

\definecolor{darkblue}{rgb}{0.0, 0.0, 0.6}

\hypersetup{colorlinks=true, urlcolor=darkblue, linkcolor=darkblue, citecolor=darkblue}

% Replace ``true'' flag below with ``false'' if you can't compile the bibliography properly using BibLaTeX/biber
\newboolean{USEBIBLATEX} \setboolean{USEBIBLATEX}{true}
\ifthenelse{\boolean{USEBIBLATEX}}{  
  \DeclareRobustCommand{\VAN}[2]{#1}
  \usepackage[hyperref=true, backend=biber, citestyle=authoryear, style=alphabetic, sortlocale=en_US, url=false, doi=false, eprint=true, firstinits=true, maxbibnames=4]{biblatex}                      
  \addbibresource{a1inv-k2.bib}
  \renewbibmacro*{volume+number+eid}{\ifentrytype{article}{\- \iffieldundef{volume}{}{{\bf\printfield{volume}},}\iffieldundef{number}{}{ no.~\printfield{number},}}}
  \renewbibmacro{in:}{\ifentrytype{article}{}{\printtext{\bibstring{in}\intitlepunct}}}
  \newbibmacro{string+doi}[1]{\iffieldundef{doi}{\iffieldundef{url}{#1}{\href{\thefield{url}}{#1}}}{\href{http://dx.doi.org/\thefield{doi}}{#1}}}
  \DeclareFieldFormat{title}{\usebibmacro{string+doi}{\mkbibemph{#1}}}
  \DeclareFieldFormat[article, inproceedings, inbook, thesis]{title}{\usebibmacro{string+doi}{\mkbibquote{#1}}}
  \DeclareFieldFormat[online]{title}{``#1''}
  \renewcommand*{\bibfont}{\footnotesize}
}{}

\oddsidemargin 5mm
\marginparwidth 5mm
\topmargin 0mm
\textheight 225mm
\textwidth 165mm
\headheight 0mm
\headsep 10mm
\footskip 5mm

\newlist{thmlist}{enumerate}{1} \setlist[thmlist]{label=(\roman{thmlisti}), ref=\thethm.(\roman{thmlisti}),noitemsep} \Crefname{thmlisti}{Theorem}{Theorems}
\newlist{lemlist}{enumerate}{1} \setlist[lemlist]{label=(\roman{lemlisti}), ref=\thelem.(\roman{lemlisti}),noitemsep} \Crefname{lemlisti}{Lemma}{Lemmas}

\newtheorem{thm}{Theorem}
\Crefname{thm}{Theorem}{Theorems}
\numberwithin{equation}{section}

\newtheorem{lem}{Lemma}
\numberwithin{lem}{section}
\Crefname{lemma}{Lemma}{Lemmas}

\newtheorem{cor}[lem]{Corollary}
\Crefname{cor}{Corollary}{Corollaries}

\newtheorem{prop}[lem]{Proposition}
\Crefname{prop}{Proposition}{Propositions}

\newtheorem*{thm*}{Theorem}
\newtheorem*{lemma*}{Lemma}

\theoremstyle{definition}

\newtheorem{dfn}[lem]{Definition}
\Crefname{dfn}{Definition}{Definitions}
\newtheorem{example}[lem]{Example}
\Crefname{example}{Example}{Examples}

\theoremstyle{remark}

\newtheorem{rem}[lem]{Remark}
\Crefname{rem}{Remark}{Remarks}

\DeclareMathOperator{\UU}{U}
\DeclareMathOperator{\LL}{L}
\DeclareMathOperator{\FU}{FU}
\DeclareMathOperator{\GL}{GL}
\DeclareMathOperator{\PGL}{PGL}
\DeclareMathOperator{\Sp}{Sp}
\DeclareMathOperator{\SL}{SL}
\DeclareMathOperator{\St}{St^G}
\DeclareMathOperator{\EE}{E}
\DeclareMathOperator{\OO}{O}
\DeclareMathOperator{\Ep}{\mathcal E}
\DeclareMathOperator{\WW}{W}
\DeclareMathOperator{\HH}{H}
\DeclareMathOperator{\XX}{X}
\DeclareMathOperator{\RR}{\mathcal R}
\DeclareMathOperator{\Lie}{Lie}
\DeclareMathOperator{\Cent}{Cent}
\DeclareMathOperator{\rad}{rad}
\DeclareMathOperator{\Norm}{Norm}
\DeclareMathOperator{\Hom}{Hom}
\DeclareMathOperator{\Spec}{Spec}
\DeclareMathOperator{\Ga}{{\mathbf G}_a}
\DeclareMathOperator{\Gm}{{\mathbf G}_m}
\DeclareMathOperator{\Ker}{ker\,}
\DeclareMathOperator{\coker}{coker\,}
\DeclareMathOperator{\im}{im\,}
\DeclareMathOperator{\Aut}{Aut}
\DeclareMathOperator{\Out}{Out}
\DeclareMathOperator{\End}{End}
\DeclareMathOperator{\Map}{Map\,}
\DeclareMathOperator{\Dyn}{Dyn}
\DeclareMathOperator{\Sym}{Sym}
\DeclareMathOperator{\Gal}{Gal}
\DeclareMathOperator{\lev}{lev}
\DeclareMathOperator{\der}{der}
\DeclareMathOperator{\res}{res}
\DeclareMathOperator{\tr}{tr}
\DeclareMathOperator{\crank}{c-rank}
\DeclareMathOperator{\rank}{rank}
\DeclareMathOperator{\ZZ}{{\mathbb Z}}
\DeclareMathOperator{\QQ}{{\mathbb Q}}
\DeclareMathOperator{\NN}{{\mathbb N}}
\DeclareMathOperator{\PP}{{\mathbb P}}
\DeclareMathOperator{\et}{\text{\it \'et}}
\DeclareMathOperator{\fppf}{\text{\it fppf}}
\DeclareMathOperator{\FF}{{\mathbb F}}
\DeclareMathOperator{\La}{{\mathcal L}}

\newcommand{\id}{\mathrm{id}}
\newcommand{\Stb}{\mathrm{St}}
\newcommand{\Aff}{\mathbb{A}}
\newcommand{\Pro}{\mathbb{P}}
\newcommand{\pprime}{\mathfrak{p}}
\newcommand{\ad}{\mathrm{ad}}
\newcommand{\scl}{{sc}}
\newcommand{\ha}{{\widetilde{\alpha}}}
\newcommand{\qs}{\mathrm{qs}}
\newcommand{\eps}{\varepsilon}
\newcommand{\st}{\scriptstyle}
\newcommand{\ds}{\displaystyle}
\newcommand{\xx}{\hbox{${\bf (XX^{-1})}$}\ }
\newcommand{\ee}{\hbox{${\bf (E)}$}}
\newcommand{\Nu}{{\mathrm N}}
\newcommand{\Mu}{{\mathrm M}}
\newcommand{\rA}{\mathsf{A}}
\newcommand{\rB}{\mathsf{B}}
\newcommand{\rC}{\mathsf{C}}
\newcommand{\rD}{\mathsf{D}}
\newcommand{\rE}{\mathsf{E}}
\newcommand{\rF}{\mathsf{F}}
\newcommand{\rG}{\mathsf{G}}

\let\l\left
\let\r\right
\let\semir\ltimes
\let\semil\rtimes

\begin{document}

\selectlanguage{english}

\title{$\mathbb{A}^1$-invariance for unstable $K_2$}

\author{S. Sinchuk}
\address{Department of Mathematics and Mechanics, St. Petersburg State University,
St. Petersburg, Russia}
\email{sinchukss@gmail.com}

\author{A. Stavrova}
\address{Department of Mathematics and Mechanics, St. Petersburg State University,
St. Petersburg, Russia}
\email{anastasia.stavrova@gmail.com}

\author{A. Lavrenov}
\address{Mathematisches Institut der Universit\"at M\"unchen, Theresienstr. 39, D-80333 M\"unchen}
\email{avlavrenov@gmail.com}

\subjclass[2010]{19C09, 19C20, 14L15, 20G35}
\keywords{Chevalley group, non-stable $K_2$-functor, Steinberg group}

\maketitle

%\begin{abstract}
%To be written
%\end{abstract}

\section{Aim of the paper}

The aim of this text is to prove that the non-stable $K_2$-functors $K_2^G$, where $G$ is a simply
connected Chevalley group
of suitable type, satisfy $\Aff^1$-invariance on regular rings $R$ containing a field $k$, that is,
\begin{equation}\label{eq:A1-main}
K_2^G(R[t])=K_2^G(R).
\end{equation} As a corollary, we should easily deduce that
$$K_2^G(R)=KV_2^G(R),$$
where $KV_2^G(R)$ is the Karoubi--Villamayor $K$-functor associated to $G$. This functor originates from~\cite{J}.
The above equality by e.g.~\cite[Corollary 4.3.3]{AHW15} implies that $\pi_1^{\Aff^1}(G)(R)=K_2^G(R)$, i.e. we obtain an explicit presentation
for the $\Aff^1$-fundamental group of $G$ in the sense of Morel--Voevodsky.

"Suitable type"{} here means that we consider only the cases where we know the centrality of $K_2$,
or at least the Quillen-Suslin lgp. Some intermediate steps can be proved in larger generality.

Essentially, we need to prove that $K_2^G(k[x_1,\ldots,x_n])=K_2^G(k)$. (Then~\eqref{eq:A1-main} follows by standard geometric methods.)
There are two models: Tulenbaev's proof for the $\SL_n$ case~\cite{Tu} and Stavrova's proof for $K_1^G$~\cite{St-poly}.
Tulenbaev~\cite{Tu} uses stabilization of the $K_2$-functor,
and the good properties of the limit=algebraic $K$-theory.
In~\cite{St-poly} stabilization is not used. However, the key steps
of both proofs are the same: the case of $R=k$ (hidden somewhere around~\cite[p. 140]{Tu}, or, respectively,
~\cite[Theorem 3.1]{St-poly}); Quillen-Suslin lgp; $\Pro^1$-gluing (see~\cite[Theorem 5.1]{Tu} or~\cite[Theorem 1.1]{St-poly}).

\subsection{The case $R=k$}
We consider the case of $K_2^G(k[t])$ vs. $K_2^G(k)$.
(As in the $\Aff^1$-invariance of $K_1^G$, this case should be used to deduce that $K_2^G(k[t_1,\ldots,t_n])=K_2^G(k)$.)

In Tulenbaev's framework, it follows from stabilization.
However, the equality $K_2^G(k[t])=K_2^G(k)$ is sort of known for all groups.
Namely, in~\cite[Theorem 5.1]{W-k[t]}: let $k$ be an infinite field and let $G$ be a connected reductive
group over $k$. Then the inclusion $k\hookrightarrow k[t]$ induces an
isomorphism
$$H_\bullet(G(k),\mathbb{Z})\stackrel{\cong}{\longrightarrow} H_\bullet(G(k[t]),\mathbb{Z}),$$
if the order of the fundamental group of $G$ is invertible in $k$.
Once we know that the homology $H_2$ coincides with $K_2^G$ (on both sides), this gives the result.
It would be nice to check Wendt's proof; maybe, discuss it in a seminar?

Is it necessary to know the centrality in order to show that $K_2^G$ coincides with $H_2$?
If yes, then we probably know it for $K_2^G(k)$ even for isotropic groups~\cite{Deo}, but only for the good
Chevalley groups for $K_2^G(k[t])$.
We should try to understand what is proved in~\cite[Proposition 5.3]{VW} using only the universality of
the Steinberg group. This may be useful. The paper is unpublished, so again everything should be double-checked
if you want to refer to it.

\subsection{Plan of the proof}
\begin{enumerate}
\item\label{plan:3t} ({\bf done}, see \cref{thm:3t}). Let $R$ be a local ring. Show that
$$\St(R[t^{\pm 1}])=i_+(\St(R[t]))i_-(\St(R[t^{-1}]))i_+(\St(R[t]))$$
Here $i_\pm$ denote the natural homomorphisms into $\St(R[t^{\pm 1}])$.

\item\label{plan:k[t]} ({\bf done}, see \cref{thm:k[t]}). Show that $K_2^G(k[t])=K_2^G(k)$.

\item\label{plan:k-intersect} ({\bf done}).
Consequently, $\St(k[t])\to\St(k[t^{\pm 1}])$ is injective and $\St(k[t])\cap\St(k[t^{-1}])=\St(k)$ inside
$\St(k[t^{\pm 1}])$. Also, $K_2^G(k)=K_2^G(k[t^{\pm 1}])$.

The first claim uses~\eqref{plan:k[t]}. The second claim follows from~\eqref{plan:k[t]} and~\eqref{plan:3t}.

\item\label{plan:QSlgp} ({\bf done} for split ACDE, see \cref{thm:lg-k2}). Prove Quillen-Suslin lgp for $K_2^G$.

\item\label{plan:Zglu} (Zariski gluing) for any commutative ring $A$ and any non-nilpotent $f,g\in A$ such that
$A=fA+gA$, the sequence of pointed sets
$$1\longrightarrow K_2^G(A)\xrightarrow{\st g\mapsto (g,g)} K_2^G(A_f)\times K_2^G(A_g)
\xrightarrow{\st (g_1,g_2)\mapsto g_1{g_2}^{-1}} K_2^G(A_{fg})$$
is exact. The proof is usually almost the same as for~\eqref{plan:QSlgp}.

(Remark. It seems that we need this property only for $A=R[t]$ and $f,g$ non-constant polynomials.)

\item\label{plan:S-lemma} ($S$-lemma)
Let $A$ be a commutative ring, $S$ a multiplicative subset of $A$.
If $$K_2^G(A[X_1,\ldots,X_n])=K_2^G(A)$$ for some $n\ge 1$, then
$K_2^G(A_S[X_1,\ldots,X_n])=K_2^G(A_S)$ as well.

This should be easy; see~\cite[Lemma 3.6]{Abe}.

\item\label{plan:Nglu} (Nisnevich gluing) Assume that $B$ is a subring of a commutative ring $A$, and let
$h\in B$ be a non-nilpotent element. Denote by $F_h:A\to A_h$ the localization homomorphism.

(i) If $Ah+B=A$, i.e. the natural map $B\to A/Ah$ is surjective, then for any $x\in \St(A_h)$ there exist
$y\in \St(A)$ and $z\in \St(B_h)$ such that
$x=F_h(y)z$.

(ii) If moreover $Ah\cap B=Bh$, i.e. $B/Bh\to A/Ah$ is an isomorphism, and $h$ is not a zero divisor in $A$, then
the sequence of pointed sets
$$
K_2^G(B)\xrightarrow{\st g\mapsto (F_h(g),g)} K_2^G(B_h)\times K_2^G(A)\xrightarrow{\st (g_1,g_2)\mapsto g_1F_h(g_2)^{-1}}
K_2^G(A_h)
$$
is exact.

This should use something from the proof of~\eqref{plan:QSlgp} or~\eqref{plan:Zglu}; see~\cite[Lemma 3.4]{St-poly}.

\item\label{plan:P1} ($\Pro^1$-gluing) Let $A$ be any commutative ring. Show that the sequence of pointed sets
$$
1\longrightarrow K_2^G(A)\xrightarrow{\st g\mapsto (g,g)} K_2^G(A[t])\times K_2^G(A[t^{-1}])
\xrightarrow{\st (g_1,g_2)\mapsto g_1{g_2}^{-1}} K_2^G(A[t,t^{-1}])
$$
is exact.

This should use~\eqref{plan:3t}--\eqref{plan:QSlgp}.

\begin{enumerate}
 \item Prove \cref{lem:TulDiagram} for CDE. In Tulenbaev's paper this lemma invokes ``another presentation''.
  I am pretty confident that this result can be demonstrated for $\Phi=\rD_\ell, \rE_\ell$ using the same ``amalgamation`` technique as in the proof of lgp. 
 \item {\bf (done)} Prove \cref{cor:TulInj}.
 \item Prove \cref{prop:TulKernSurject}. This is hard. Have no idea how this can be proved at the moment. 
 \item Prove $\Pro^1$-glueing using all the above facts.
\end{enumerate}

\item\label{plan:[]f} (Main corollary of $\Pro^1$-gluing) Let $A$ be any commutative ring, and let $f\in A[t]$ be a monic polynomial. Show that
$K_2^G(A[t])\to K_2^G(A[t]_f)$ is injective.

The proof uses~\eqref{plan:Zglu} and~\eqref{plan:P1}.

\item\label{plan:k(t)} Prove that $K_2^G(k(t))=K_2^G(k)$. This may be a bit tricky; I will think if we can get rid of it. I don't think Tulenbaev uses it.
Unsure whether this is true (?). This is true for $SK_1$ but in view of Milnor's theorem can not hold for $K_1$ and $K_2$.

\item\label{plan:k[tn]} Prove that
$$K_2^G(k[t_1,\ldots,t_n])=K_2^G(k).$$

If we strictly follow the pattern of $K_1^G$, this uses~\eqref{plan:[]f},~\eqref{plan:k[t]}, and
~\eqref{plan:k(t)}. There may be other ways.

\item \label{plan:final} Final result: let $R$ be a regular ring containing a field $k$. Then $K_2^G(R[t])=K_2^G(R)$.

This uses~\eqref{plan:QSlgp},~\eqref{plan:k[tn]} and~\eqref{plan:Nglu}.
\end{enumerate}

%%%%%%%%%%%%%%%%%%%%%%%%%%%%%%%%%%%%%%%%%%%%%%%%%%%%%%%%%%%%%%%%%%%%%%%%%%%%%%%%%%%%%%%%%%%%
%%%%%%%%%%%%%%%%%%%%%%%%%%%%%%%%%%%%%%%%%%%%%%%%%%%%%%%%%%%%%%%%%%%%%%%%%%%%%%%%%%%%%%%%%%%%
\section{Steinberg groups of Chevalley groups: preliminaries}

Definition, functoriality, "congruence subgroups"{} $\St(\Phi,R,I)$ versus $\ker(\St(\Phi,R)\to\St(\Phi,R/I))$.

%The map $\St(R)\to E(R)$ is bijective on subgroups of the form $U_\Psi(R)$, where $\Psi$ is a
%unipotent set of roots.

\subsection{Tulenbaev's map}
The following property of linear Steinberg groups was discovered for the first time by Tulenbaev (see~\cite[Lemma 2.3]{Tu}).

\begin{dfn} \label{def:tep}
Let $R$ be arbitrary commutative ring and let $a \in R$ be any nonnilpotent element.
We say that the Steinberg group functor $\St$ satisfies {\it Tulenbaev's excision property} if there exists a map $T$ 
which completes the canonical arrows in the diagram below to a commutative diagram.
\[\begin{tikzcd} \St(R[t], tR[t]) \arrow[rr] \arrow[dr, "\lambda_a"] & & \St(n, R\ltimes tR_a[t], tR_a[t]) \\
 & \St(n, R_a[t], tR_a[t]) \arrow[ur, dotted, "T"] & \end{tikzcd}\] \end{dfn}

%%%%%%%%%%%%%%%%%%%%%%%%%%%%%%%%%%%%%%%%%%%%%%%%%%%%%%%%%%%%%%%%%%%%%%%%%%%%%%%%%%%%%%%%%%%%
%%%%%%%%%%%%%%%%%%%%%%%%%%%%%%%%%%%%%%%%%%%%%%%%%%%%%%%%%%%%%%%%%%%%%%%%%%%%%%%%%%%%%%%%%%%%
\section{Decomposition theorems for \texorpdfstring{$\St(A[t^{\pm 1}])$}{St(A[t, t\textminussuperior\textonesuperior])} and \texorpdfstring{$\St\bigl(A((t))\bigr)$}{St(A((t)))}.}

\begin{lem}
 Let $(R,m)$ be a local ring, and let $G$ be a simply connected simple group over $R$
of isotropic rank $\ge 2$. Let $i_+:\St(R[t])\to\St(R[t^{\pm 1}])$ and $i_-:\St(R[t^{-1}]\to
\St(R[t^{\pm 1}])$ be the natural homomorphisms.
Then
$$
i_+\bigl(\St(m\cdot R[t])^{\St(R[t])}\bigr)i_-\bigl(\St(R[t^{-1}]\bigr)=i_-\bigl(\St(R[t^{-1}]\bigr)
i_+\bigl(\St(m\cdot R[t])^{\St(R[t])}\bigr)
$$
inside $\St(R[t^{\pm 1}])$.
\end{lem}
\begin{proof}
This is proved exactly as~\cite[Lemma 5.12]{St-poly}.
\end{proof}


\begin{thm}\label{thm:3t}
Let $R$ be a local ring, and let $G$ be a simply connected simple group over $R$
of isotropic rank $\ge 2$. Let $i_+:\St(R[t])\to\St(R[t^{\pm 1}])$ and $i_-:\St(R[t^{-1}]\to
\St(R[t^{\pm 1}])$ be the natural homomorphisms.
Then
$$
\St(R[t^{\pm 1}])=i_+(\St(R[t]))i_-(\St(R[t^{-1}]))i_+(\St(R[t])).
$$
\end{thm}
\begin{proof}
This is proved exactly as~\cite[Theorem 5.1]{St-poly}.
\end{proof}


%%%%%%%%%%%%%%%%%%%%%%%%%%%%%%%%%%%%%%%%%%%%%%%%%%%%%%%%%%%%%%%%%%%%%%%%%%%%%%%%%%%%%%%%%%%%
%%%%%%%%%%%%%%%%%%%%%%%%%%%%%%%%%%%%%%%%%%%%%%%%%%%%%%%%%%%%%%%%%%%%%%%%%%%%%%%%%%%%%%%%%%%%
\section{Quillen-Suslin lgp, Zariski gluing, Nisnevich gluing, $S$-lemma}

\begin{thm}\label{thm:lg-k2}
Assume that the Steinberg group functor $\St$ satisfies Tulenbaev's excision property (cf.~\cref{def:tep}).
Then an element $g \in \St(R[t], tR[t])$ is trivial if and only if its image in $\St(R_M[t], tR_M[t])$ is trivial for all maximal ideals $M \trianglelefteq R$.
\end{thm}
\begin{proof}
In the case $\Phi=\rC_\ell$, $\ell\geq 3$ the assertion of the theorem is the main result of \cite{La15}.
For a simply laced $\Phi$ of rank $\geq 3$ this can be proved by the same token as \cite[Theorem~2]{S15} if one uses a stronger variant of Tulenbaev's lemma proved in the appendices below (see \cref{cor:tulmap}).
\end{proof}

%%%%%%%%%%%%%%%%%%%%%%%%%%%%%%%%%%%%%%%%%%%%%%%%%%%%%%%%%%%%%%%%%%%%%%%%%%%%%%%%%%%%%%%%%%%%
%%%%%%%%%%%%%%%%%%%%%%%%%%%%%%%%%%%%%%%%%%%%%%%%%%%%%%%%%%%%%%%%%%%%%%%%%%%%%%%%%%%%%%%%%%%%
\section{The case of \texorpdfstring{$K_2^G(k[t])$}{K\texttwoinferior(k[t])} and some corollaries}
Let $G = G(\Phi, -)$ be the simply-connected Chevalley---Demazure group scheme of type $\Phi$.

Recall that for $\alpha\in\Phi$, $\varepsilon\in R^*$ the semisimple root elements $h_\alpha(\varepsilon)$ are defined as $h_\alpha(\varepsilon)=w_\alpha(\varepsilon)w_\alpha(-1)$.

\begin{thm} \label{thm:k[t]}
Let $k$ be a field. Assuming that $\Phi$ is irreducible of rank at least $2$ the following facts are true.
\begin{thmlist}
 \item \label{satz1} For $A=k, k[t]$ the group $K_2^G(A)$ is generated by elements of the form 
  $$h_\alpha(uv) h_\alpha(u)^{-1} h_\alpha(v)^{-1},\ u,v\in k^*.$$
 \item �s a consequence, one has $K_2^G(k[t])=K_2^G(k).$
\end{thmlist}
\end{thm}
\begin{proof}
 See \cite[Satz~1]{Re75} and the corollary after it.
\end{proof}

\begin{cor}\label{cor:k[t]inj}
Let $G$, $k$ be as in \cref{thm:k[t]}. 
Then $\St(k[t])\to\St(k[t^{\pm 1}])$ is injective and $\St(k[t])\cap\St(k[t^{-1}])=\St(k)$ inside $\St(k[t^{\pm 1}])$.
\end{cor}
\begin{proof}
Clearly, $g\in\ker(\St(k[t])\to\St(k[t^{\pm 1}]))$ implies $g\in K_2^G(k[t])$. Since $K_2^G(k[t])=K_2^G(k)$,
and  there is a section $K_2^G(k[t^{\pm 1}])\to K_2^G(k)$, the map is injective.
Second claim: take $g\in \St(k[t])\cap\St(k[t^{-1}])$.
Then the image $\phi(g)$ belongs to $E(k)=E(k[t])\cap E(k[t^{-1}])$, and
after adjusting $g$ by an element of $\St(k)$, we can assume that $g\in K_2^G(k[t])\cap K_2^G(k[t^{-1}])$. Hence
$g\in K_2^G(k)\subseteq\St(k)$.
\end{proof}

\begin{cor}
Let $G$, $k$ be as in \cref{thm:k[t]}. Then $K_2^G(k[t^{\pm 1}])=K_2^G(k)$.
\end{cor}
\begin{proof}
We use \cref{thm:3t}.
Take $g\in K_2^G(k[t^{\pm 1}])$,
then $g=x_1yx_2$, $x_i\in \St(k[t]))$, $y\in \St(k[t^{-1}]))$.
Since $E(k[t])\cap E(k[t^{-1}])=E(k)$,
we have $y\in\St(k)K_2^G(k[t^{-1}])=\St(k)$ and $x_1x_2\in\St(k)$. That is, $g\in K_2^G(k)$.
\end{proof}

Denote by $\WW(\Phi, R)$ the subgroup of $\Stb(\Phi, R)$ generated by all elements $w_\alpha(\varepsilon)$, $\varepsilon\in R^*$.
The following result is a Steinberg level analogue of the well-known Bruhat decomposition.
\begin{cor} \label{cor:bruhat}
 For $\Phi$ of rank $\geq 2$ and a field $k$ one has $$\Stb(\Phi, k) = \UU(\Phi^+, k) \cdot \WW(\Phi, k) \cdot \UU(\Phi^+, k).$$
\end{cor}
\begin{proof}
 From \cref{satz1} it follows that $K_2(\Phi, k)$ is contained in $\WW(\Phi, k)$, therefore the reqired assertion follows from \cite[Theorem~4]{St-lect}.
 For a direct proof in the linear case see also \cite[\S~2.3A]{HOM}.
\end{proof}



%%%%%%%%%%%%%%%%%%%%%%%%%%%%%%%%%%%%%%%%%%%%%%%%%%%%%%%%%%%%%%%%%%%%%%%%%%%%%%%%%%%%%%%%%%%%
%%%%%%%%%%%%%%%%%%%%%%%%%%%%%%%%%%%%%%%%%%%%%%%%%%%%%%%%%%%%%%%%%%%%%%%%%%%%%%%%%%%%%%%%%%%%
\section{\texorpdfstring{$\Pro^1$}{P\textonesuperior}-gluing}
\subsection{Preliminaries}

Let $R$ denote arbitrary commutative local ring with the maximal ideal $m$ and the residue field $k$.
Consider the following commutative diagram of groups.
\[\begin{tikzcd} 
C_+ \ar[r, hookrightarrow] \ar[dddd, "k^+", twoheadrightarrow] & \St(R[t], m[t]) \ar[rd, twoheadrightarrow] \ar[rr, "\mu^+"] \ar[dd, "j^+_1"] & & \ar[dd, "i^+_1"] \St(R[t]) \ar[r, "p^+", twoheadrightarrow] & \St(k[t]) \ar[dd, equal] \\    
 &  & \mathrm{Ker}(p^+) \ar[ur, hookrightarrow] \ar[dd, "\overline{j^+_1}" near end] & &  \\    
 & \St(R[t] + m[t^{-1}], m[t^{\pm 1}]) \ar[rd, twoheadrightarrow] \ar[rr, "\mu^{+\varepsilon}" near start] \ar[dd, "j^+_2"] & & \ar[dd, "i^+_2"] \St(R[t]+m[t^{-1}]) \ar[r, "p^{+\varepsilon}", twoheadrightarrow] & \St(k[t]) \ar[dd, hookrightarrow, "\overline{i_2^+}"] \\     
 &  & \mathrm{Ker}(p^{+\varepsilon}) \ar[ur, hookrightarrow] \ar[dd, "\overline{j^+_2}" near end, twoheadrightarrow] & &  \\    
C_{\pm} \ar[r, hookrightarrow] & \St(R[t^{\pm 1}], m[t^{\pm 1}]) \ar[uurr, bend right=15, dashrightarrow, "\varphi"' near end] \ar[rd, twoheadrightarrow] \ar[rr, "\mu^\pm"' near start] &  & \St(R[t^{\pm 1}]) \ar[r, "p^\pm", twoheadrightarrow] & \St(k[t^{\pm 1}]) \\
 &  & \mathrm{Ker}(p^\pm) \ar[ur, hookrightarrow] & &  
\end{tikzcd}\]
By the argument from A. Stavrova's 08/11/15 letter one also has that $\overline{j^+_2}$ is surjective.

\begin{lem} \label{lem:TulDiagram} There exists a map $\varphi$ such that $ i_2^+ \varphi = \mu^\pm$. \end{lem}
\begin{proof}
 Compare with \cite[Lemma~3.2]{Tu}.  
\end{proof}

\begin{prop} \label{prop:TulKernSurject} The map $k^+$ is surjective. \end{prop}
\begin{proof}
 Compare with \cite[Proposition~4.1]{Tu}. This should invoke \cref{lem:TulDiagram}. 
\end{proof}

\begin{cor} \label{cor:TulInj} The map $i^+_2$ is injective. \end{cor}
\begin{proof}
 Follows from the above lemmata by a simple diagram chasing (cf.~\cite[Cor.~4.2]{Tu}).
 
 Indeed, let $g \in \St(R[t] + m[t^{-1}])$ be an element of $\Ker(i_2^+)$.
 By \cref{cor:k[t]inj} $g$ also lies in $\Ker(p^{+\varepsilon})$ and hence comes from some $\widetilde{g} \in \St(A[t]+m[t^{-1}], m[t^{\pm 1}])$ via $\mu^{+\varepsilon}$.
 Since $j^+_2(\widetilde{g})$ lies in $C_\pm$ by \cref{prop:TulKernSurject} it comes from some $\widehat{g} \in C_+$ via $k^+$.
 It remains to notice that $g = \mu^{+\varepsilon}(\widetilde{g}) = i_1^+\mu^+(\widehat{g}) = i_1^+(1) = 1$, as claimed. 
\end{proof}

\subsection{Main result}

The following lemma is an analog of~\cite[Proposition 4.3 (a)]{Tu}.

\begin{lem}
Let $R$ be a local ring, $k=R/m$, and let $k$, $G$ be as in \cref{thm:k[t]}.
The natural homomorphism $\St(R[t])\to\St(R[t^{\pm 1}])$ is injective.
\end{lem}
\begin{proof}
Let $I$ be the maximal ideal of $R$,
$l=R/I$, and consider the natural maps $\rho:\St(R[t,t^{-1}])\to \St(l[t,t^{-1}])$,
$\rho_+:\St(R[t])\to \St(l[t])$, $\rho_-:\St(R[t^{-1}])\to\St(l[t^{-1}])$.
Take $x\in \ker(\St(R[t])\to\St(R[t^{\pm 1}])$.
By the field case \cref{cor:k[t]inj} one has
$\rho_+(x)=1,$ hence $x\in\St(I\cdot R[t])^{\St(R[t])}$.

???????
\end{proof}

\begin{lem}
Let $G$, $k$ be as in \cref{thm:k[t]}.
Let $(R,m)$ be a local ring such that $R/m=k$.
Then
$$\St(R[t])\cap \St(R[t^{-1}])=\St(R)$$
inside $\St(R[t^{\pm 1}])$.
\end{lem}
\begin{proof}
??????
\end{proof}

\begin{thm}
Let $A$ be any commutative ring. Then the sequence of pointed sets
$$1\longrightarrow K_2^G(A)\xrightarrow{\st g\mapsto (g,g)} K_2^G(A[t])\times K_2^G(A[t^{-1}])
\xrightarrow{\st (g_1,g_2)\mapsto g_1{g_2}^{-1}} K_2^G(A[t,t^{-1}])$$
is exact.
\end{thm}
\begin{proof}
Follows from the above lemmas.
\end{proof}

\begin{cor}
 Let $A$ be any commutative ring, and let $f\in A[t]$ be a monic polynomial. Show that
$K_2^G(A[t])\to K_2^G(A[t]_f)$ is injective.
\end{cor}
\begin{proof}
????
\end{proof}

%%%%%%%%%%%%%%%%%%%%%%%%%%%%%%%%%%%%%%%%%%%%%%%%%%%%%%%%%%%%%%%%%%%%%%%%%%%%%%%%%%%%%%%%%%%%
%%%%%%%%%%%%%%%%%%%%%%%%%%%%%%%%%%%%%%%%%%%%%%%%%%%%%%%%%%%%%%%%%%%%%%%%%%%%%%%%%%%%%%%%%%%%
\begin{appendices}

\section{Linear Steinberg group in rank $3$}

The main goal of this subsection is to show that Tulenbaev's \cite[Lemma~2.3]{Tu} remains valid for the linear Steinberg group of rank $\geq 3$.
In order to do this we will need yet another presentation for the relative linear Steinberg group (cf. \cite[Definitions~3.3 and 3.7]{S15}).
\begin{dfn}
 The relative Steinberg group $\Stb^*(n,R, I)$ is the group defined by the following two
 families generators and four families of relations.
 \begin{itemize}
  \item Generators:
  \begin{enumerate}
  \item $X^1(u, v)$, where $u \in \EE(n,R) \cdot e_1$, $v\ \in I^n$ such that $v^t \cdot u = 0$;
  \item $X^2(u, v)$, where $u \in I^n$, $v \in \EE(n,R) \cdot e_1$ such that $v^t \cdot u = 0$.
 \end{enumerate}
  Notice that $\phi$ maps both $X^1(u, v)$ and $X^2(u, v)$ to $T(u, v) = e + u \cdot v^t \in \EE(n, R, I)$.
  \item Relations:
  \begin{enumerate}
  \item $X^1(u, v) \cdot X^1(u, w) = X^1(u, v+w)$, $u \in \EE(n,R) \cdot e_1$, $v, w \in I^n$;
  \item $X^2(u, v) \cdot X^2(w, v) = X^2(u+w, v)$, $u, w \in I^n$, $v \in \EE(n,R) \cdot e_1$;
  \item ${}^{X^\sigma(u^2, v^2)} \! X^\tau(u^1, v^1) = X^\tau(T(u^2, v^2) \cdot u^1, T(v^2, u^2)^{-1} \cdot v^1)$, $\sigma, \tau = 1,2$;
  \item $X^1(g \cdot e_1, g^* \cdot be_2) = X^2(g \cdot be_1, g^* \cdot e_2)$ where $b\in I$ and $g^* = {g^t}^{-1}$ denotes the contragradient matrix.
 \end{enumerate}
 \end{itemize}
\end{dfn}

\begin{lem}
 The groups $\Stb^*(n, R, I)$ and $\Stb(n, R, I)$ are isomorphic.
\end{lem}
\begin{proof}
 {\bf TODO:}
\end{proof}

The next step of the proof is to is construct certain elements in $\Stb(n, R)$ similar to Tulenbaev's elements $X_{u,v}(a)$ see~\cite[\S~1]{Tu}.

Let $v\in R^n$ be a column.
Denote by $O(v)$ the submodule of $R^n$ consisting of all columns $w$ such that $w^t \cdot v = 0$.
A column $w\in R^n$ is called \emph{$v$-decomposable} if it can be presented as a finite sum $w = \sum\limits_{i=1}^p w^i$ such that each $w^i$ has at least two zero entries and $v^t \cdot w^i = 0$.
Denote by $D(v)$ the submodule of $O(v)$ consisting of all $v$-decomposable columns.
For a column $v\in R^n$ denote by $I(v)$ the ideal of $R$ spanned by its entries $v_1,\ldots, v_n$.

Let $u,v,w\in R^n$ be columns such that $w^tv=0$.
It is easy to check (cf.~\cite[Lemma~3.2]{Ka}) that
$$(uv)\cdot w = \sum_{i<j}w_{ij},\ \text{where}\ w_{ij} = (w_iu_j - w_ju_i)(v_je_i - v_ie_j)\in{}\!A^n.$$
The above decomposition is called the \emph{canonical} decomposition of $(uv)\cdot w$.
In particular, this shows that the column $a\cdot w$ is always $v$-decomposable for $a\in I(v)$, $w \in O(v)$, i.\,e. $I(v) \cdot O(v) \subseteq D(v)$.
It is also straightforward to check that $D(v)\subseteq D(bv)$, $b \cdot D(v) \subseteq D(v)$ for $b \in R$.

Denote by $B^1$ the subset of $R^n \times R^n \times R$ consisting of triples $(u, v, a)$ such that $v^t \cdot u = 0$, $v \in D(u)$, $a \in I(u)$.
Denote by $B^2$ the set consisting of triples $(v, u, a)$ such that $(u, v, a) \in B^1$.

\begin{lem} \label{lem:Zfacts}
Assume that $n \geq 4$.
One can define two families of elements $Z^\tau(u, v, a)$, $\tau=1,2$ of the group $\Stb(n, R)$ parametrized by $(u, v, a) \in B^\tau$ satisfying the following properties:
 \begin{enumerate}
  \item $\phi(Z^\tau(u, v, a)) = e + uav^t \in \EE(n, R)$, $(u,v,a) \in B^\tau$;
  \item $Z^{1}(u, v + w, a) = Z^{1}(u, v, a) \cdot Z^{1}(u, w, a)$;
  \item $Z^{2}(v + w, u, a) = Z^{2}(v, u, a) \cdot Z^{2}(w, u, a)$;
  \item for $\tau=1,2$ and $b \in R$ if $(u,vb,a), (ub, v, a) \in B^\tau$ then one has
   $$Z^\tau(u,vb, a) = Z^\tau(u, v, ab) = Z^\tau (ub, v, a);	$$
  \item ${}^{g}\! Z^{\tau}(u, v, a) = Z^{\tau}(\phi(g) \cdot u, \phi(g)^* \cdot v, a)$, $\tau = 1,2$, $g \in \St(n, R)$.
 \end{enumerate}
\end{lem}
\begin{proof}
One constructs the elements $Z^1(u,v,a)$ in exactly the same way as Tulenbaev constructs his elements $X_{u,v}(a)$ (see definitions preceding~\cite[Lemma~1.2]{Tu}).
Indeed, set \begin{equation} Z^1(v, w, a) = \prod\limits_{k=1}^p X(v, a \cdot w^k), \quad Z^2(w, v, a) = \prod\limits_{k=1}^p X(a \cdot w^k, v). \end{equation}
where $X(u, v)$ denotes the elements defined by Tulenbaev before~\cite[Lemma~1.1]{Tu}.

The correctness of this definition and all the assertions of the lemma (with the exception of the last one in the case $n=4$) can be proved by the same token as in~\cite[Lemma~1.3]{Tu}.
%TODO: Add more details
\end{proof}

For the rest of this section $a$ denotes a nonnilpotent element of $R$ and $\lambda_a \colon R \rightarrow R_a$ is the morphism of principal localization at $a$.
\begin{lem} \label{lem:rk3rels} For any $g \in \EE(n, R_a)$ there exist $u, v \in R^n$ and sufficiently large natural numbers $k$, $m$ such that the following facts hold:
\begin{enumerate}
 \item $\lambda_a(u) = g \cdot a^k e_1$, $\lambda_a(v) = g^* \cdot a^k e_2$ and $u^t \cdot v = 0$;
 \item $(u, v, a^m) \in B^1 \cap B^2$;
 \item for $b \in R$ divisible by some sufficiently large power of $a$ one has
             $$Z^1(u, b \cdot v, a^m) = Z^2(b \cdot u, v, a^m).$$
\end{enumerate}
\end{lem}
\begin{proof}
It is straightforward to choose $u$ and $v$ satisfying the first requirement of the lemma.
We can even choose $u$, $v$ in such a way that $u \in D(v)$ and $v \in D(u)$.
Indeed, notice that $I(u) = a^{k_1}$, $I(v) = a^{k_2}$ for some natural $k_1$, $k_2$ hence for $u' = a^{k_2} \cdot u$ and $v' = a^{k_1} \cdot v$ one has
$$u' \in I(v) \cdot O(v) \subseteq D(v) \subseteq D(v'),\quad v' \in I(u) \cdot O(u) \subseteq D(u) \subseteq D(u'),$$
as required.

In fact, we can also choose two extra columns $x, y \in R^n$ and a large natural $p$ in such a way that vectors $u,v,x,y$ additionally satisfy the following properties
\begin{equation*} \lambda_a(x) = g^* \cdot a^k e_3,\ \lambda_a(y) = g \cdot a^k e_3,\ y^t \cdot x = a^p \in R, \end{equation*}
\begin{equation*} u^t \cdot x = 0,\ u^t \cdot v = 0,\ y^t \cdot v = 0, \end{equation*}
\begin{equation*} (u, x, a^m) \in B^1,\ (y, v, a^m) \in B^2. \end{equation*}

Now direct computation using \cref{lem:Zfacts} shows that
 \begin{multline*}
 Z^2(a^{m+p}b \cdot u, v, a^m) = Z^2(b \cdot (e+a^m \cdot ux^t)y, (e-a^m \cdot xu^t)v,a^m) \cdot Z^2(-by,v,a^m) = \\
  = [Z^1(u, x, a^m), Z^2(b \cdot y, v, a^m)] = \\
    = Z^1(u,x,a^m) \cdot Z^1((e+a^mb \cdot yv^t)u,-(e- a^mb \cdot vy^t)x,a^m) = Z^1(u,a^{m+p}b \cdot v,a^m), \qedhere
 \end{multline*}
hence the third assertion of the lemma follows.
\end{proof}

\begin{cor}\label{cor:tulmap}[Tulenbaev's lemma] For $n \geq 4$ there is a map $T_n$ so that the following diagram commutes.

\end{cor}
\begin{proof} Follows from \cref{lem:rk3rels} by the same token as in \cite[Lemma~2.3]{Tu}. \end{proof}

\end{appendices}

\ifthenelse{\boolean{USEBIBLATEX}}{
\DeclareRobustCommand{\VAN}[2]{#2}
\printbibliography}{
\renewcommand{\refname}{References}
\begin{thebibliography}{MMMM}

\bibitem[A]{Abe} E. Abe,
{\it Whitehead groups of Chevalley groups over polynomial rings},
Comm. Algebra {\bf 11} (1983), 1271--1307.

\bibitem[AHW]{AHW15} A. Asok, M. Hoyois, M. Wendt,
{\it Affine representability results in $A^1$-homotopy theory II: principal bundles and homogeneous spaces},
2015~\href{http://arxiv.org/abs/1507.08020}{arXiv:1507.08020}.

\bibitem[Deo]{Deo} V. V. Deodhar,
{\it On central extensions of rational points of algebraic groups},
Amer. J. Math. {\bf 100} (1978), 303--386.

\bibitem[J]{J} J.F. Jardine,
{\it On the homotopy groups of algebraic groups},
J. Algebra {\bf 81} (1983), 180--201.

\bibitem[Si]{S15} S. Sinchuk,
{\it \href{http://dx.doi.org/10.1016/j.jpaa.2015.08.003}{On centrality of $\mathrm{K}_2$ for {C}hevalley groups of type $\rE_\ell$}},
J. Pure Appl. Algebra {\bf 220} (2016), 857--875.

\bibitem[St14]{St-poly} A. Stavrova,
{\it Homotopy invariance of non-stable $K_1$-functors},
J. K-Theory {\bf 13} (2014), 199--248.

\bibitem[St15]{St-serr} A. Stavrova,
{\it \href{http://dx.doi.org/10.4153/CJM-2015-035-2}{Non-stable $K_1$-functors of multiloop groups}},
Canad. J. Math. (2015), Online First.

\bibitem[Su]{Sus} A.A.~Suslin,
\emph{On the structure of the special linear group over polynomial rings},
Math. USSR Izv. {\bf 11} (1977), 221--238.

\bibitem[Tul82]{Tu}
M.~S. Tulenbaev, \emph{The {S}teinberg group of a polynomial ring},
Mat. Sb. (N.S.) \textbf{117(159)} (1982), no.~1, 131--144.

\bibitem[Ka]{Ka}
W.~van~der~Kallen, \emph{\href{http://dx.doi.org/10.1016/1385-7258(77)90026-9}{Another presentation for {S}teinberg groups}},
Indag. Math. {\bf 80}, (1977), no.~4, 304--312.

\bibitem[La15]{La15}
A.~Lavrenov, \emph{Local-global principle for symplectic $K_2$}
{\bf to appear}.

\bibitem[Wen14]{W-k[t]}
M.~Wendt, \emph{\href{https://dx.doi.org/10.4310/MRL.2014.v21.n6.a15}{On homology of linear groups over {$k[t]$}}},
Math. Res. Lett. \textbf{21} (2014), no.~6, 1483--1500.

\bibitem[VW]{VW} K. V\"olkel, M. Wendt,
\emph{On $\Aff^1$-fundamental groups of isotropic reductive groups},
2012~\href{http://arxiv.org/abs/1207.2364}{arXiv:1207.2364}.
\end{thebibliography}

} %END \ifthenelse

\end{document}
