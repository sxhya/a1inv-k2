The main goal of this subsection is to show that Tulenbaev's \cite[Lemma~2.3]{Tu} remains valid for the linear Steinberg group of rank $\geq 3$.
In order to do this we will need yet another presentation for the relative linear Steinberg group (cf. \cite[Definitions~3.3 and 3.7]{S15}).
\begin{dfn}
 The relative Steinberg group $\Stb^*(n,R, I)$ is the group defined by the following two
 families generators and four families of relations.
 \begin{itemize}
  \item Generators:
  \begin{enumerate}
  \item $X^1(u, v)$, where $u \in \EE(n,R) \cdot e_1$, $v\ \in I^n$ such that $v^t \cdot u = 0$;
  \item $X^2(u, v)$, where $u \in I^n$, $v \in \EE(n,R) \cdot e_1$ such that $v^t \cdot u = 0$.
 \end{enumerate}
  Notice that $\phi$ maps both $X^1(u, v)$ and $X^2(u, v)$ to $T(u, v) = e + u \cdot v^t \in \EE(n, R, I)$.
  \item Relations:
  \begin{enumerate}
  \item $X^1(u, v) \cdot X^1(u, w) = X^1(u, v+w)$, $u \in \EE(n,R) \cdot e_1$, $v, w \in I^n$;
  \item $X^2(u, v) \cdot X^2(w, v) = X^2(u+w, v)$, $u, w \in I^n$, $v \in \EE(n,R) \cdot e_1$;
  \item ${}^{X^\sigma(u^2, v^2)} \! X^\tau(u^1, v^1) = X^\tau(T(u^2, v^2) \cdot u^1, T(v^2, u^2)^{-1} \cdot v^1)$, $\sigma, \tau = 1,2$;
  \item $X^1(g \cdot e_1, g^* \cdot be_2) = X^2(g \cdot be_1, g^* \cdot e_2)$ where $b\in I$ and $g^* = {g^t}^{-1}$ denotes the contragradient matrix.
 \end{enumerate}
 \end{itemize}
\end{dfn}

\begin{lem}
 The groups $\Stb^*(n, R, I)$ and $\Stb(n, R, I)$ are isomorphic.
\end{lem}
\begin{proof}
 {\bf TODO:}
\end{proof}

The next step of the proof is to is construct certain elements in $\Stb(n, R)$ similar to Tulenbaev's elements $X_{u,v}(a)$ see~\cite[\S~1]{Tu}.

Let $v\in R^n$ be a column.
Denote by $O(v)$ the submodule of $R^n$ consisting of all columns $w$ such that $w^t \cdot v = 0$.
A column $w\in R^n$ is called \emph{$v$-decomposable} if it can be presented as a finite sum $w = \sum\limits_{i=1}^p w^i$ such that each $w^i$ has at least two zero entries and $v^t \cdot w^i = 0$.
Denote by $D(v)$ the submodule of $O(v)$ consisting of all $v$-decomposable columns.
For a column $v\in R^n$ denote by $I(v)$ the ideal of $R$ spanned by its entries $v_1,\ldots, v_n$.

Let $u,v,w\in R^n$ be columns such that $w^tv=0$.
It is easy to check (cf.~\cite[Lemma~3.2]{Ka}) that
$$(uv)\cdot w = \sum_{i<j}w_{ij},\ \text{where}\ w_{ij} = (w_iu_j - w_ju_i)(v_je_i - v_ie_j)\in{}\!A^n.$$
The above decomposition is called the \emph{canonical} decomposition of $(uv)\cdot w$.
In particular, this shows that the column $a\cdot w$ is always $v$-decomposable for $a\in I(v)$, $w \in O(v)$, i.\,e. $I(v) \cdot O(v) \subseteq D(v)$.
It is also straightforward to check that $D(v)\subseteq D(bv)$, $b \cdot D(v) \subseteq D(v)$ for $b \in R$.

Denote by $B^1$ the subset of $R^n \times R^n \times R$ consisting of triples $(u, v, a)$ such that $v^t \cdot u = 0$, $v \in D(u)$, $a \in I(u)$.
Denote by $B^2$ the set consisting of triples $(v, u, a)$ such that $(u, v, a) \in B^1$.

\begin{lem} \label{lem:Zfacts}
Assume that $n \geq 4$.
One can define two families of elements $Z^\tau(u, v, a)$, $\tau=1,2$ of the group $\Stb(n, R)$ parametrized by $(u, v, a) \in B^\tau$ satisfying the following properties:
 \begin{enumerate}
  \item $\phi(Z^\tau(u, v, a)) = e + uav^t \in \EE(n, R)$, $(u,v,a) \in B^\tau$;
  \item $Z^{1}(u, v + w, a) = Z^{1}(u, v, a) \cdot Z^{1}(u, w, a)$;
  \item $Z^{2}(v + w, u, a) = Z^{2}(v, u, a) \cdot Z^{2}(w, u, a)$;
  \item for $\tau=1,2$ and $b \in R$ if $(u,vb,a), (ub, v, a) \in B^\tau$ then one has
   $$Z^\tau(u,vb, a) = Z^\tau(u, v, ab) = Z^\tau (ub, v, a);	$$
  \item ${}^{g}\! Z^{\tau}(u, v, a) = Z^{\tau}(\phi(g) \cdot u, \phi(g)^* \cdot v, a)$, $\tau = 1,2$, $g \in \St(n, R)$.
 \end{enumerate}
\end{lem}
\begin{proof}
One constructs the elements $Z^1(u,v,a)$ in exactly the same way as Tulenbaev constructs his elements $X_{u,v}(a)$ (see definitions preceding~\cite[Lemma~1.2]{Tu}).
Indeed, set \begin{equation} Z^1(v, w, a) = \prod\limits_{k=1}^p X(v, a \cdot w^k), \quad Z^2(w, v, a) = \prod\limits_{k=1}^p X(a \cdot w^k, v). \end{equation}
where $X(u, v)$ denotes the elements defined by Tulenbaev before~\cite[Lemma~1.1]{Tu}.

The correctness of this definition and all the assertions of the lemma (with the exception of the last one in the case $n=4$) can be proved by the same token as in~\cite[Lemma~1.3]{Tu}.
%TODO: Add more details
\end{proof}

For the rest of this section $a$ denotes a nonnilpotent element of $R$ and $\lambda_a \colon R \rightarrow R_a$ is the morphism of principal localization at $a$.
\begin{lem} \label{lem:rk3rels} For any $g \in \EE(n, R_a)$ there exist $u, v \in R^n$ and sufficiently large natural numbers $k$, $m$ such that the following facts hold:
\begin{enumerate}
 \item $\lambda_a(u) = g \cdot a^k e_1$, $\lambda_a(v) = g^* \cdot a^k e_2$ and $u^t \cdot v = 0$;
 \item $(u, v, a^m) \in B^1 \cap B^2$;
 \item for $b \in R$ divisible by some sufficiently large power of $a$ one has
             $$Z^1(u, b \cdot v, a^m) = Z^2(b \cdot u, v, a^m).$$
\end{enumerate}
\end{lem}
\begin{proof}
It is straightforward to choose $u$ and $v$ satisfying the first requirement of the lemma.
We can even choose $u$, $v$ in such a way that $u \in D(v)$ and $v \in D(u)$.
Indeed, notice that $I(u) = a^{k_1}$, $I(v) = a^{k_2}$ for some natural $k_1$, $k_2$ hence for $u' = a^{k_2} \cdot u$ and $v' = a^{k_1} \cdot v$ one has
$$u' \in I(v) \cdot O(v) \subseteq D(v) \subseteq D(v'),\quad v' \in I(u) \cdot O(u) \subseteq D(u) \subseteq D(u'),$$
as required.

In fact, we can also choose two extra columns $x, y \in R^n$ and a large natural $p$ in such a way that vectors $u,v,x,y$ additionally satisfy the following properties
\begin{equation*} \lambda_a(x) = g^* \cdot a^k e_3,\ \lambda_a(y) = g \cdot a^k e_3,\ y^t \cdot x = a^p \in R, \end{equation*}
\begin{equation*} u^t \cdot x = 0,\ u^t \cdot v = 0,\ y^t \cdot v = 0, \end{equation*}
\begin{equation*} (u, x, a^m) \in B^1,\ (y, v, a^m) \in B^2. \end{equation*}

Now direct computation using \cref{lem:Zfacts} shows that
 \begin{multline*}
 Z^2(a^{m+p}b \cdot u, v, a^m) = Z^2(b \cdot (e+a^m \cdot ux^t)y, (e-a^m \cdot xu^t)v,a^m) \cdot Z^2(-by,v,a^m) = \\
  = [Z^1(u, x, a^m), Z^2(b \cdot y, v, a^m)] = \\
    = Z^1(u,x,a^m) \cdot Z^1((e+a^mb \cdot yv^t)u,-(e- a^mb \cdot vy^t)x,a^m) = Z^1(u,a^{m+p}b \cdot v,a^m), \qedhere
 \end{multline*}
hence the third assertion of the lemma follows.
\end{proof}

\begin{cor}\label{cor:tlpA3} For $G=G(\rA_3, -)$ the Steinberg group functor $\St(R)$ satisfies Tulenbaev lifting property~\ref{def:tlp}.

\end{cor}
\begin{proof} Follows from \cref{lem:rk3rels} by the same token as in \cite[Lemma~2.3]{Tu}. \end{proof}
