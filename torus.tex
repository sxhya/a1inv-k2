Our notation and conventions follows~\cite[\S~4]{VavWE}.
Let $\Phi$ be an irreducible root system with some fixed basis of simple roots $\Pi = \{\alpha_1, \ldots, \alpha_\ell\}$.
We denote by $\Phi^\vee$ the \emph{dual root system of $\Phi$} consisting of vectors $\alpha^\vee = 2\alpha/(\alpha, \alpha)$, $\alpha\in \Phi$.
As usual, $P(\Phi^\vee)$ denotes the lattice spanned by the \emph{fundamental weights $\varpi_i$}.
Recall that $\varpi_i$ are uniquely determined by relations $\langle\varpi_i, \alpha_j^\vee \rangle = (\varpi_i, \alpha_j) = \delta_{ij}.$

Notice that for $\varpi \in P(\Phi^\vee)$ and $\beta \in \ZZ \Phi$ one has $(\varpi, \beta) \in \ZZ$.
Consequently, for $\varepsilon \in R^*$ and $\varpi \in P(\Phi^\vee)$ the identity $\chi_{\varpi, \varepsilon}(\beta) = \varepsilon ^ {(\varpi, \beta)}$
gives a well-defined character $\chi_{\varpi, \varepsilon} \in \Hom(\ZZ \Phi, R^*)$.

Consider the action of $H=\Hom(\ZZ \Phi, R^*)$ on the set of generators $\mathcal{X}_{\Phi, R}$ of the Steinberg group $\St(R)$ defined by
\begin{equation} \chi \cdot x_\alpha(\xi) = x_\alpha(\chi(\alpha) \cdot \xi),\ \chi \in H,\ \alpha\in \Phi,\ \xi \in R. \end{equation}
Since $\chi$ is a character, the above action preserves the set of Steinberg relations $\mathcal{R}_{\Phi, R}$ and,
thus, gives a well-defined action of $H$ on $\St(R)$.

\begin{example} The principal example which motivates the above construction is as follows.
Let $A$ be a ring, take $R = A[t, t^{-1}]$ to be the ring of Laurent polynomials over $A$ and let $\alpha_i \in \Pi$ be some simple root.
Since $t \in R^*$ we can consider the automorphism $\sigma_i$ of $\Stb(\Phi, R)$ given by $\sigma_i = \chi_{\varpi_i, t}$.
It is easy to see that 
\begin{equation}\sigma_i(x_\alpha(\xi)) = x_\alpha(t^{m_i(\alpha)} \cdot \xi),\end{equation}
where $m_k(\alpha)$ denotes the coefficient in the expansion of $\alpha$ in $\Pi$, i.\,e. $\alpha = \sum m_k(\alpha) \alpha_k$.
\end{example}

Our next goal is to define an analogue of $\sigma_i$ for the group $\St(A[t])$.
Obviously, we cannot expect such map to be automorphism or even be defined on the whole group $\St(A[t])$.
However, it turns out that in some cases it is still possible to define it on a certain subgroup of $\St(A[t])$.

\begin{lem} \label{lem:sigma}
 Assume that $A$ is a local commutative ring, $G = G(\Phi, -)$ and $(\Phi, i)$ is as follows:
 \begin{itemize}
  \item $\Phi = \rA_\ell$, $\ell \geq 3$, $i=1$;
  \item $\Phi$ is classical of rank $\ell\geq 4$, $i=1$;
  \item $\Phi = \rE_\ell$, $\ell=6,7,8$, $i=\ell$.
 \end{itemize}

 Then there exists subgroups $N_i^+$, $N_i^-$ of $\St(A[t])$ and a homomorphism $\sigma_i' \colon N_i^+ \to N_i^-$ 
 compatible with the action of $\sigma_i$ i.\,e. such that the following diagram commutes:
 \[\begin{tikzcd} N_i^+ \arrow[r, "\lambda_t^*"] \arrow[d, "\sigma_i'"]          &  \St(A[t, t^{-1}]) \ar[d, "\sigma_i"] \\
                  N_i^- \arrow[r, "\lambda_t^*"] &  \St(A[t, t^{-1}]) \end{tikzcd}\]

\end{lem}
We first prove the assertion of the lemma in the linear case and then deduce all the other cases from it.

In the case when $m_i(\widetilde{\alpha})=1$ the subgroups $N_i^\pm$ can be constructed explicitly as follows.
For $\alpha_i\in\Pi$ consider the subgroup $P_i^+$ (resp. $P_i^-$) of $\St(A)$ generated by $x_\alpha(\xi)$ for $\xi \in A$, $\alpha\in\Sigma_i^+ \cup \Delta_i$ (resp. $\alpha\in\Sigma_i^- \cup \Delta_i$).
Now, let $N_{i}^+$ (resp. $N_i^-$) be the subgroup consisting of $g \in \St(A[t])$ such that $g(0) \in P_i^+$ (resp. $g(0) \in P_i^-$).

\begin{proof}[Proof in the case $\Phi=\rA_\ell$, $\ell\geq 3$.]

Denote by $i_\ell$ the natural map $\Stb^{G(\Delta_1)}(A)\to \Stb^{G(\Phi)}(A)$. Notice that $\Delta_1 \cong \rA_{\ell-1}$.
By the Levi decomposition $P_1^+$ is isomorphic to $\UU(\Sigma_1, A) \rtimes \im(i_\ell)$.

The stable rank of $A$ equals $1$ hence from the injective stability theorem for $K_2$ (see~\cite[Theorem~4.1]{ST76}) it follows that the map $i_\ell$ is injective for $\ell \geq 3$
and that $P_1^+ \cong \UU(\Sigma_1, A) \rtimes \Stb^{G(\Delta_1)}(A)$.

By \cref{item:st-semi} $\St(A[t]) = \St(A) \ltimes \St(A[t], tA[t])$ hence $N_1^\pm = P_1^\pm \ltimes \St(A[t], tA[t])$.
Define $\sigma_1'$ on the generators $X^1(u, v)$, $X^2(v, u)$ of $\St(A[t], tA[t])$ ($u \in \EE(\ell+1, A[t])\cdot e_1$, $v\in {tA[t]}^{\ell+1}$) by
$$ \sigma_1'(X^1(u, v)) = Z^1()$$
\end{proof}
