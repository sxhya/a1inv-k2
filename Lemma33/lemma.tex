\documentclass[oneside, 10pt]{amsart}
\usepackage[breaklinks=true]{hyperref}
\usepackage[hyperref=true, backend=bibtex, firstinits=true, citestyle=numeric-comp, sortlocale=en_US, url=false, doi=false, eprint=true, maxbibnames=4]{biblatex}
\usepackage[capitalize]{cleveref}
\usepackage[matrix,arrow,curve]{xy}
\usepackage{enumitem}

\addbibresource{lemma.bib}
\renewbibmacro*{volume+number+eid}{\ifentrytype{article}{\- \iffieldundef{volume}{}{Vol.~\printfield{volume},}\iffieldundef{number}{}{ No.~\printfield{number},}}}
\renewbibmacro{in:}{\ifentrytype{article}{}{\printtext{\bibstring{in}\intitlepunct}}}
\newbibmacro{string+doi}[1]{\iffieldundef{doi}{\iffieldundef{url}{#1}{\href{\thefield{url}}{#1}}}{\href{http://dx.doi.org/\thefield{doi}}{#1}}}
\DeclareFieldFormat[article, inproceedings, inbook, book, thesis]{title}{\usebibmacro{string+doi}{\mkbibquote{#1}}}
\renewcommand*{\bibfont}{\footnotesize}


\newcommand{\affinize}[1]{{\widetilde{#1}^{(1)}}}
\DeclareMathOperator{\aff}{aff}
\DeclareMathOperator{\St}{St}
\DeclareMathOperator{\rk}{rk}
\newcommand{\XX}[1]{\mathcal{X}_{#1}}
\newcommand{\RR}[1]{\mathcal{R}_{#1}}
\newtheorem{proposition}{Proposition}
\Crefname{prop}{Proposition}{Propositions}
\newtheorem{theorem}{Theorem}
\Crefname{thm}{Theorem}{Theorems}
\newtheorem{corollary}{Corollary}
\newtheorem{lemma}{Lemma}
\newlist{lemlist}{enumerate}{1} \setlist[lemlist]{label={\rm(\arabic{lemlisti})}, ref=\thelemma.(\arabic{lemlisti}),noitemsep} \Crefname{lemlisti}{Lemma}{Lemma}

\theoremstyle{remark}

\newtheorem{rem}[lemma]{Remark}
%\Crefname{rem}{Remark}{Remarks}
\begin{document}   
Let $A$ be arbitrary commutative unital ring and $\mathfrak{m}$ be its ideal.
Denote by $B$ the ring $\mathfrak{m}[t^{-1}] + A[t]$ with the obvious $\mathbb{Z}$-grading.
Clearly, the $k$-th homogeneous component $B_k$ of $B$ equals $A\cdot t^k$ for $k \geq 0$ or $B_k = \mathfrak{m} \cdot t^k$ for $k<0$.

For $n \geq 1$ consider the following collections of generators:
\[\XX{n} = \{ x_\alpha(\xi) \mid \xi \in B_k,\text{ for } k\leq n\}. \]
It is obvious that $\XX{n} \subseteq \XX{n+1}$, denote by $\XX{\infty}$ the union of all $\XX{n}$'s.

Let $\Phi$ be a simply-laced irreducible root system of rank $\geq 3$.
Denote by $G_m$ the group presented by generators $\mathcal{X}_m$ and the set of relations $\RR{m}$ consisting of the following
 three families of relations:
\begin{align}
 \label{eq:am} \tag{$a_m$} x_\alpha(\xi) x_\alpha(\eta) & = x_\alpha(\xi+\eta),&  \xi,\eta\in B_k,\ k\leq m;&\\
 \label{eq:bm} \tag{$b_m$} [x_\alpha(\xi), x_{\alpha'}(\eta))] &  = 1, & \text{in the case $\alpha+\alpha'\not\in\Phi\cup\{0\}$,}\\
 \nonumber                                                     &       & \xi \in B_k,\ \eta \in B_l,\ l,k\leq m;\\
 \label{eq:cm} \tag{$c_m$} [x_\alpha(\xi), x_{\alpha'}(\eta)] & = x_{\alpha+\alpha'}(N_{\alpha,\alpha'}\xi\eta), & \text{in the case $\alpha+\alpha'\in \Phi$,}\\
 \nonumber                                                    &                                                  & \xi \in B_k,\ \eta \in B_l,\ l, k, l+k \leq m.
\end{align}

The aim of this subsection is to the following analogue of~\cite[Lemma~3.3]{Tu}. 
\begin{lemma}\label{lem:tul3.3}
 For all simply-laced root systems of rank $\geq 3$ the natural inclusion of generators induces an isomorphism $G_1 \to \St(\Phi, B).$
\end{lemma}
Unfortunately, the lemma cannot be proved by a slight modification of the original short argument of Tulenbayev.
 %for even in the linear case it works only for $\mathrm{rk}(\Phi)\geq 4$, and, in particular, is unsuitable in the case $\Phi=\mathsf{D}_\ell$.
Instead, we have to employ a variant of much longer argument of U.~Rehmann and C.~Soul{\'e}, see~\cite{R75,RS76}.

%Denote by $F(S)$ the free group on the generating set $S$.
%It is clear that $\RR{n}\subseteq\RR{n+1}$, moreover $\langle \RR{n} \rangle = \langle \RR{\infty}\rangle \cap F(\XX{n})$.
%well, not so clear, actually :)
To prove~\cref{lem:tul3.3} it suffices to show that the natural map $\theta_m \colon G_m \to G_{m+1}$ is an isomorphism for $m\geq 1$.
Indeed, this would give the required isomorphism $G_1 \cong \St(\Phi, B)$ since $\langle \XX{\infty} \mid \RR{\infty} \rangle$ is evidently isomorphic to $\St(\Phi, B)$.

\begin{rem}
There are two reasons why we can't simply refer to~\cite{R75} or~\cite{RS76} for the proof of~\cref{lem:tul3.3}.
The first is that Rehmann and Soul{\'e} prove that $G_m \to G_{m+1}$ is an isomorphism for $m\geq 2$, while we want to establish this fact starting from $m = 1$.
The second is that they prove their results only in the special case $B = A[t]$, moreover, they make assumptions about the ground ring $A$
(it is assumed that $A=k$ is a field in~\cite{R75} and that $A=\mathbb{Z}$ in~\cite{RS76}).
% (this can be replaced by the weaker assumption that $A$ is additively generated by its units).
\end{rem}

We will use the following commutator identities (cf.~\cite[H1]{R75}):
\begin{align}
% \label{eq:H1i}   [a, bc] = [a, b]\cdot {}^b[a, c] = [a, b] \cdot [b, [a, c]] \cdot [a, c];& \\
 \label{eq:H1ii}  [ab, c] = {}^a[b, c] \cdot [a,c];&\\ %= [a,[b,c]]\cdot [b,c] \cdot [a,c];&\\
 \label{eq:H1iii} [a,c]   = 1    \text{ implies } [a, [b,c]] = [[a,b],{}^bc].&
\end{align}

The following lemma is reminiscent of~\cite[Proposition 1.1]{R75} and~\cite[Proposition~3.2.2]{RS76}.
\begin{lemma}
 Suppose $m \geq 1$.
 Let $\alpha, \beta, \alpha', \beta'$ be such that $\alpha + \beta = \alpha' + \beta'$.
 Assume, moreover, that $\xi \in B_k$, $\xi' \in B_{k'}$, $\eta \in B_l$, $\eta' \in B_{l'}$ are such that 
  $N_{\alpha, \beta} \xi \eta = N_{\alpha', \beta'}\xi' \eta'$ for some $k,k',l,l'\leq m$ satisfying $k+l=k'+l' = m+1$.
 Then in the group $G_m$ the following relations hold:
 \begin{align}
  \label{item:i} [x_\alpha(\xi), x_\beta(\eta)] &= [x_{\alpha'}(\xi'), x_{\beta'}(\eta')] & \\
  \label{item:ii} [[x_\gamma(\zeta), [x_\alpha(\xi), x_\beta(\eta)]] & = 1 & \text{in the case } \gamma\in\{\alpha, \beta, \alpha + \beta\},\\
  \nonumber                                     &                          & \text{$\zeta \in B_{k''}$, $k''\leq m$.}
 \end{align}
\end{lemma}
\begin{proof}
 Notice that $k+l = m+1$, $k, l\leq m$ imply $k,l>0$, hence $B_i= t^i \cdot A$ for $i=k,k',l,'l'$.
 Therefore, we can repeat the argument of~\cite[Proposition 1.1]{R75} verbatim.
\end{proof}

To prove that $\theta_m$ is an isomorphism we construct the missing generators of $G_{m+1}$ in $G_m$ and then show that they satisfy relations $\RR{m+1}$.

For every $\xi \in B_{m+1}$ and $\alpha\in \Phi$ there exist $\xi' \in B_m$ and $\alpha'\in \Phi$ such that $\xi = t\xi'$ and $\alpha-\alpha'\in\Phi$.
Thus, we can make the following definition:
\begin{equation} \label{eq:defmpn} x_\alpha(\xi) := [x_{\alpha-\alpha'}(N_{\alpha-\alpha',\alpha'} \xi'), x_{\alpha'}(t)],\end{equation} 
and from~\eqref{item:i} its correctness (i.\,e. the independence of the choice of $\alpha'$) follows.

We now turn to the verification of relations $(a_{m+1})$, $(b_{m+1})$, $(c_{m+1})$.
Notice first that \eqref{eq:H1ii} and~\eqref{item:ii} immediately imply $(a_{m+1})$ and hence $(b_{m+1})$ in the special case $\alpha=\alpha'$.

To verify relations $(c_{m+1})$ it suffices to show that
\begin{equation}
\label{eq:verify-bmp1} [x_\alpha(\xi), x_{\alpha'}(at^{m+1})] = [x_\alpha(t\xi), x_{\alpha'}(at^m)],\ a \in A,\ \xi \in B_k,\ k\leq 0.
\end{equation}
We can find root subsystem $\Psi \subseteq \Phi$ of type $\mathsf{A}_3$ containing the roots $\alpha, \alpha'$.
Choose basis $\{\alpha,\beta,\gamma\}$ of $\Psi$ so that $\alpha'=\beta$ and the Dynkin diagram of $\Psi$ looks as follows:
\begin{equation}\label{eq:diagA3} \xymatrix{\circ_\alpha \ar@{-}[r] & \circ_\beta \ar@{-}[r] & \circ_\gamma}.\end{equation}
Let us verify~\eqref{eq:verify-bmp1}:
\begin{align*}
   [x_\alpha(\xi), x_\beta(at^{m+1})] = [x_\alpha(\xi), [x_{\beta + \gamma}(t), x_{-\gamma}(a't^m)]]&  \text{ by~\eqref{eq:defmpn} for a suitable $a'\in A$} \\ 
 = [x_{\alpha+\beta+\gamma}(\epsilon t\xi), {}^{x_{\beta+\gamma}(t)}x_{-\gamma}(a't^m)]             &  \text{ by~\eqref{eq:H1iii}, for $\epsilon=N_{\alpha, \beta+\gamma}$} \\
 = {}^{x_{\beta+\gamma}(t)}[x_{\alpha+\beta+\gamma}(\epsilon t\xi), x_{-\gamma}(a't^m)]             &  \text{ by $(b_1)$} \\
 = {}^{x_{\beta+\gamma}(t)}[x_{\alpha+\beta+\gamma}(\epsilon t^2\xi), x_{-\gamma}(a't^{m-1})]       &  \text{ by~\eqref{item:i} if $k=0$ or~\eqref{eq:cm} if $k <0$} \\
 = [[x_\alpha(t\xi), x_{\beta+\gamma}(t)], {}^{x_{\beta+\gamma}(t)} x_{-\gamma}(a't^{m-1})]         &  \text{ by~$(b_2)$, $(c_2)$ or by~\eqref{item:ii},\eqref{eq:defmpn} if $m=1$} \\
 = [x_\alpha(t\xi), x_\beta(at^m)]                                                                  &  \text{ by~\eqref{eq:H1iii}.}
\end{align*}

Now, let us show that relations $(b_{m+1})$ hold. 
Without loss of generality we may also assume $k \leq l=m+1$, thus $\eta = bt^{m+1}$ for some $b\in A$.

\begin{enumerate}
\item \label{case:1} First consider the case $k \leq 0$. There are two further subcases.
 \begin{enumerate}
  \item \label{case:1a} Case $\alpha \not \perp \alpha'$. 
  As before, without loss of generality we may assume that $\alpha, \alpha'$ are contained in a root system $\Psi$ of type $\mathsf{A}_3$,
   whose simple roots are labeled as in~\eqref{eq:diagA3} and, moreover, $\alpha'=\alpha + \beta$.
  Using~\eqref{eq:H1iii} and~\eqref{eq:bm} we get (recall that $\xi \in B_k$):
   \begin{multline} \nonumber
   [x_\alpha(\xi), x_{\alpha+\beta}(bt^{m+1})] = [x_\alpha(\xi), [x_{\alpha+\beta+\gamma}(bt^m), x_{-\gamma}(t)]] = \\   
  = [[x_\alpha(\xi), x_{\alpha+\beta+\gamma}(bt^m)], {}^{x_{\alpha+\beta+\gamma}(bt^m)}\!x_{-\gamma}(t)] = 1.
  \end{multline}
  \item Case $\alpha \perp \alpha'$. As before, we set $\alpha' = \gamma$ and use~\eqref{eq:H1iii},\eqref{eq:bm} and~\eqref{eq:cm}:
  \begin{multline} \nonumber
   [x_\alpha(\xi), x_{\gamma}(bt^{m+1})] = [x_\alpha(\xi), [x_{\beta+\gamma}(bt^m), x_{-\beta}(t)]] = \\   
  = [[x_\alpha(\xi), x_{\beta+\gamma}(bt^m)], {}^{x_{\beta+\gamma}(bt^m)}\!x_{-\beta}(t)] = {}^{x_{\beta+\gamma}(bt^m)}\![x_{\alpha+\beta+\gamma}(bt^m\xi), x_{-\beta}(t)] = 1.
  \end{multline}
 \end{enumerate} 
  
\item \label{case:2} Now assume $1 \leq k \leq m+1$. 
At first we want to prove $(b_{m+1})$ only in the special case $\xi=t^k$.
We proceed by induction on $k$ starting with $k=1$.
Again, there are two cases.
\begin{enumerate}
 \item \label{case:2a} Case $\alpha\not\perp\alpha'$ is handled similarly to the Case~\eqref{case:1a} with the only
  difference that we have to refer to the inductive assumption rather than~\eqref{eq:bm} in the case $k=m+1$.
  \begin{multline} \nonumber
   [x_\alpha(t^k), x_{\alpha+\beta}(bt^{m+1})] = [x_\alpha(t^k), [x_{\alpha+\beta+\gamma}(bt^m), x_{-\gamma}(t)]] = \\   
  = [[x_\alpha(t^k), x_{\alpha+\beta+\gamma}(bt^m)], {}^{x_{\alpha+\beta+\gamma}(bt^m)}\!x_{-\gamma}(t)] = 1.
  \end{multline}

 \item Case $\alpha\perp\alpha'$. As before, we may assume $\alpha'=\gamma$.
\begin{align*}
{}^{x_\gamma(t^k)}x_\alpha(bt^{m+1}) = {}^{x_\gamma(t^k)}[x_{-\beta}(b't^{m+1}), x_{\alpha+\beta}(1)] & \text{ by $(c_{m+1})$}\\
 = [x_{-\beta}(b't^{m+1}), {}^{x_\gamma(t^k)}x_{\alpha+\beta}(1)]                                     & \text{ by Case~\eqref{case:2a} since $\gamma\not\perp-\beta$}\\
 = [[x_{-\beta-\gamma}(b''t^{m+1-k}), x_\gamma(t^k)], {}^{x_\gamma(t^k)}x_{\alpha+\beta}(1)]          & \text{ by $(c_{m+1})$}\\
 = [x_{-\beta-\gamma}(b''t^{m+1-k)}), [x_\gamma(t^k), x_{\alpha+\beta}(1)]]                           & \text{ by~\eqref{eq:H1iii} and~\eqref{eq:bm} since $m+1-k \leq m$}\\
 = x_{\alpha}(b'''t^{m+1})                                                                            & \text{ by $(c_{m+1})$.} \end{align*}
 Usual identities for structure constants imply (cf.~\cite[p.~12]{R75}):
 \[b'''=N_{-\beta-\gamma, \alpha+\beta+\gamma} \cdot N_{\gamma, \alpha+\beta} \cdot N_{-\beta-\gamma, \gamma} \cdot N_{-\beta, \alpha+\beta} \cdot b = b, \] 
\end{enumerate}
This finishes the demonstratation of $(b_{m+1})$ in the special case $\xi=t^k$.
\item Now let us prove $(b_{m+1})$ for arbitrary $\xi = at^k$, $a\in A$ and $1\leq k\leq m+1$. Again, there are two subcases.
\begin{enumerate}
\item \label{case:3a} Case $\alpha \not\perp \alpha'$. % not possible to induct on k here :(
 \begin{align*} [x_\alpha(at^k), x_{\alpha + \beta}(bt^{m+1})] = [x_\alpha(at^k), [x_{\alpha+\beta+\gamma}(t^{m+1}), x_{-\gamma}(b)]] & \text{ by $(c_{m+1})$} \\
  = [[x_\alpha(at^k), x_{\alpha+\beta+\gamma}(t^{m+1})], {}^{x_{\alpha+\beta+\gamma}(t^{m+1})}\!x_{-\gamma}(b)] & \text{ by $(b_k)$ or Case~\eqref{case:1} if $k=m+1$} \\
  = 1 & \text{ by Case~\eqref{case:2}.} \end{align*}
\item Case $\alpha \perp \alpha'$.
 % mu        0  1 m+1
 % vu        1  0 k
 % mu'       0  1 m+1-k
 % mu - mu'  0  0 k
 % nu + mu'  1  1 m+1
 %           a  b t
The required identity follows from the following chain of equalities:
\begin{align*}
 x_{\alpha+\beta+\gamma}(abt^{m+1}) = {}^{x_{-\beta}(t^k)}\!x_{\alpha+\beta+\gamma}(abt^{m+1}) & \text{ by Case~\eqref{case:2}}\\
 = {}^{x_{-\beta}(t^k)}\![x_{\alpha+\beta}(bt^{m+1-k}), x_\gamma(at^k)] & \text{ by $(c_{m+1})$} \\
 = [x_{\alpha}(bt^{m+1}) x_{\alpha+\beta}(bt^{m+1-k}), x_\gamma(at^k)] & \text{ by $(b_k)$ or Case~\eqref{case:2} if $k=m+1$ }\\
 = {}^{x_{\alpha}(bt^{m+1})}\!x_{\alpha+\beta+\gamma}(abt^{m+1}) [x_{\alpha}(bt^{m+1}), x_\gamma(at^k)] & \text{ by~\eqref{eq:H1ii} and $(c_{m+1})$} \\
 = x_{\alpha+\beta+\gamma}(abt^{m+1}) [x_{\alpha}(bt^{m+1}), x_\gamma(at^k)] & \text{ by Case~\eqref{case:3a}.}
\end{align*}
\end{enumerate}
\end{enumerate}


\printbibliography
\end{document}
